\mode*

% Since this a solution template for a generic talk, very little can
% be said about how it should be structured. However, the talk length
% of between 15min and 45min and the theme suggest that you stick to
% the following rules:  

% - Exactly two or three sections (other than the summary).
% - At *most* three subsections per section.
% - Talk about 30s to 2min per frame. So there should be between about
%   15 and 30 frames, all told.


\section[Bootstrapping]{Bootstrapping authentication}

\subsection{What is bootstrapping?}

\begin{frame}
  \begin{block}{Bootstrapping: A hen-and-egg problem}
    \begin{itemize}
      \item Alice is not registered in our authentication system.
      \item We want to register her as a user in our system.

        \pause{}

      \item How do we know Alice is actually Alice?
      \item Since she's not registered we cannot authenticate her.
    \end{itemize}
  \end{block}

  \pause{}

  \begin{exercise}
    \begin{itemize}
      \item Any quick workarounds that comes to mind?
      \item When is this a problem and when is it not?
    \end{itemize}
  \end{exercise}
\end{frame}

\begin{frame}
  \begin{solution}[We don't care who Alice is]
    \begin{itemize}
      \item We simply set up authentication when Alice creates the account.
      \item Now we can authenticate whoever set up the account.
    \end{itemize}
  \end{solution}

  \pause{}

  \begin{example}
    This is the solution used by most web services.
  \end{example}
\end{frame}

\begin{frame}
  \begin{solution}[We care who Alice actually is]
    \begin{itemize}
      \item We can require ID checks etc.\ to set up the authentication 
        mechanisms using a helpdesk.

        \pause{}

      \item If we have address etc., then we can send the credentials via mail 
        (be it snailmail or email).

    \end{itemize}
  \end{solution}

  \pause{}

  \begin{example}[The university account]
    This is how your university account was set up.
  \end{example}
\end{frame}

\begin{frame}
  \begin{exercise}
    \begin{itemize}
      \item How is Alice authenticated when she applies for an ID?
    \end{itemize}
  \end{exercise}
\end{frame}

\begin{frame}
  \begin{example}[Signal, WhatsApp, \dots]
    \begin{itemize}
      \item The identity is a (mobile) phone number.
      \item Send a text message with a code.
    \end{itemize}
  \end{example}

  \pause{}

  \begin{remark}
    \begin{itemize}
      \item Phone provider can impersonate.
      \item Government can impersonate (forcing phone provider).
    \end{itemize}
  \end{remark}
\end{frame}

\subsection{Problems with bootstrapping}

\begin{frame}
  \begin{enumerate}
    \item Attacker \emph{intercepts} a password on account creation.
      \begin{itemize}
        \item User starts bootstrapping.
        \item Password is sent to user.
        \item Attacker grabs password.
      \end{itemize}

      \pause{}

    \item Attacker \emph{impersonates} the legitimate user.
      \begin{itemize}
        \item Attacker starts bootstrapping.
        \item User remains unaware.
        \item Service cannot distinguish attacker and user.
      \end{itemize}
  \end{enumerate}
\end{frame}

\begin{frame}
  \begin{itemize}
    \item It can be costly to manage.

    \item Sometimes it is a continuous process, if the same bootstrapping 
      procedure is also used for \emph{recovery from failure}.

    \item Make sure the system can handle forgotten, lost or aged 
      authentication means.

  \end{itemize}
\end{frame}

\subsection{Single Sign-On}

\begin{frame}
  \begin{itemize}
    \item We could let someone else who has solved the problem already do the 
      authentication for us.

    \item This way the user only needs one username and password, and he or she 
      only needs to sign in once.

    \item However, this makes the SSO provider a very attractive target.

    \item And they are forced to solve our problem anyway.

    \item The problem is, now we need to trust them to do it properly \dots
  \end{itemize}
\end{frame}

\begin{frame}
  \begin{example}[We don't care who Alice is]
    We can use Google, Facebook etc.
  \end{example}

  \begin{example}[We care who Alice is]
    We can use e.g.\ BankID.
  \end{example}

  \begin{remark}
    The SSO-service must have done bootstrapping as rigorously as we would 
    have.
  \end{remark}
\end{frame}


%%%%%%%%%%%%%%%%%%%%%%


\begin{frame}[allowframebreaks]
	\small
  \printbibliography{}
\end{frame}

