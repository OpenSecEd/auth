% $Id$
%\documentclass[handout]{beamer}
\documentclass{beamer}
\usepackage[utf8]{inputenc}
\usepackage[T1]{fontenc}
\usepackage[swedish,english]{babel}
\usepackage{url}
\usepackage{graphicx}
\usepackage{color}
\usepackage{keystroke}

\setbeamertemplate{bibliography item}[text]
\usepackage[natbib,style=alphabetic,maxbibnames=99]{biblatex}
\addbibresource{keyauth.bib}

\mode<presentation>{%
  \usetheme{Frankfurt}
  \setbeamercovered{transparent}
  \usecolortheme{seagull}
}
\setbeamertemplate{footline}{\insertframenumber}

\title{%
  Key Management and Authentication
}
\author{Daniel Bosk}
\institute[MIUN IKS]{%
  Department of Information and Communication Systems,\\
  Mid Sweden University, SE-851\,70 Sundsvall.
}
\date{\today}

%\pgfdeclareimage[height=0.65cm]{university-logo}{MU_logotyp_int_CMYK.pdf}
%\logo{\pgfuseimage{university-logo}}

\AtBeginSection[]{%
  \begin{frame}<beamer>{Overview}
		\tableofcontents[currentsection]
	\end{frame}
}

\begin{document}

\begin{frame}
  \titlepage{}
\end{frame}

\begin{frame}{Overview}
	\tableofcontents
	% You might wish to add the option [pausesections]
\end{frame}


% Since this a solution template for a generic talk, very little can
% be said about how it should be structured. However, the talk length
% of between 15min and 45min and the theme suggest that you stick to
% the following rules:  

% - Exactly two or three sections (other than the summary).
% - At *most* three subsections per section.
% - Talk about 30s to 2min per frame. So there should be between about
%   15 and 30 frames, all told.


\section{Symmetric Key Distribution}

\subsection{Symmetric Crypto}

\begin{frame}{\insertsubsectionhead}
  \begin{figure}
    \includegraphics[width=0.7\textwidth]{symmetric.eps}
    \caption{An overview of symmetric crypto.
      Image: \cite{Stallings2013nse}.
    }
  \end{figure}
\end{frame}

\subsection{Key Distribution Centre (KDC)}

\begin{frame}{\insertsubsectionhead}
  \begin{itemize}
    \item Deliver a key \(k\) from \(A\) to \(B\).
      By themselves or third party.

    \item If \(A\) and \(B\) share a key \(k\), generate a key \(k^\prime\) and 
      transmit it using \(k\):
      \(A\to B\colon E_k(k^\prime)\).

    \item Secure connection to third party \(C\), \(C\) delivers key to \(A\) 
      and \(B\).

  \end{itemize}
\end{frame}

\begin{frame}{\insertsubsectionhead}
  \begin{description}
    \item[Session Key] Temporary key used between \(A\) and \(B\).

    \item[Permanent Key] Key used to distribute session keys.

    \item[Key Distribution Centre] The central entity with which permanent keys 
      are shared and by whom session keys are generated.

  \end{description}
\end{frame}

\subsection{Authentication}

\begin{frame}{\insertsubsectionhead}
  \begin{figure}
    \includegraphics[height=0.7\textheight]{kerberos-overview.eps}
    \caption{An overview of Kerberos.
      Image: \cite{Stallings2013nse}.
    }
  \end{figure}
\end{frame}

\subsection{Kerberos IV}

\begin{frame}{\insertsubsectionhead}
  \begin{figure}
    \includegraphics[height=0.7\textheight]{kerberos-ver4.eps}
    \caption{An overview of Kerberos IV authentication dialogue.
      Image: \cite{Stallings2013nse}.
    }
  \end{figure}
\end{frame}

\begin{frame}{\insertsubsectionhead}
  \begin{figure}
    \includegraphics[height=0.7\textheight]{kerberosIV-detailed.eps}
    \caption{Kerberos IV authentication protocol.
      Image: \cite{Stallings2013nse}.
    }
  \end{figure}
\end{frame}

\subsection{Kerberos V}

\begin{frame}{\insertsubsectionhead}
  \begin{figure}
    \includegraphics[height=0.7\textheight]{kerberosV-detailed.eps}
    \caption{Kerberos V authentication protocol.
      Image: \cite{Stallings2013nse}.
    }
  \end{figure}
\end{frame}

\begin{frame}{\insertsubsectionhead}{Environmental Differences}
  \begin{itemize}
    \item Encryption system dependence.
    \item Internet protocol dependence.
    \item Byte ordering.
    \item Ticket lifetime.
    \item Authentication forwarding.
    \item Interrealm authentication.
  \end{itemize}
\end{frame}

\begin{frame}{\insertsubsectionhead}{Technical differences}
  \begin{itemize}
    \item Double encryption.
    \item Propagating Cipher Block Chaining instead of CBC.
    \item Session and subsession keys.
    \item Password attacks.
  \end{itemize}
\end{frame}


\section{Asymmetric Key Distribution}

\subsection{Asymmetric Crypto and Hash Functions}

\begin{frame}{\insertsubsectionhead}
  \begin{figure}
    \includegraphics[height=0.7\textheight]{asymmetric.eps}
    \caption{An overview of asymmetric crypto.
      Image: \cite{Stallings2013nse}.
    }
  \end{figure}
\end{frame}

\begin{frame}{\insertsubsectionhead}
  \begin{figure}
    \includegraphics[height=0.7\textheight]{mac-hash.eps}
    \caption{An overview of using hash functions for message integrity and 
      authentication.
      Image: \cite{Stallings2013nse}.
    }
  \end{figure}
\end{frame}

\subsection{Diffie--Hellman Key Exchange}

\begin{frame}{\insertsubsectionhead}
  \begin{figure}
    \includegraphics[height=0.7\textheight]{dhkx.eps}
    \caption{A schematic overview of the Diffie--Hellan Key Exchange algorithm.
      Image: \cite{Stallings2013nse}.
    }
  \end{figure}
\end{frame}

\begin{frame}{\insertsubsectionhead}
  \begin{figure}
    \includegraphics[height=0.7\textheight]{dhkx-mitm.eps}
    \caption{Schematic overview of a Man-in-the-Middle Attack.
      Image: \cite{Stallings2013nse}.
    }
  \end{figure}
\end{frame}

\subsection{Public-key Certificates}

\begin{frame}{\insertsubsectionhead}
  \begin{figure}
    \includegraphics[width=\textwidth]{cert-overview.eps}
    \caption{An overview of use of public-key certificates.
      Image: \cite{Stallings2013nse}.
    }
  \end{figure}
\end{frame}

\begin{frame}{\insertsubsectionhead}{X.509}
  \begin{figure}
    \includegraphics[height=0.7\textheight]{x509-format.eps}
    \caption{An overview of X.509 certificate format.
      Image: \cite{Stallings2013nse}.
    }
  \end{figure}
\end{frame}

\begin{frame}{\insertsubsectionhead}
  \begin{figure}
    \includegraphics[height=0.7\textheight]{sign-overview.eps}
    \caption{An overview of the digital signature process.
      Image: \cite{Stallings2013nse}.
    }
  \end{figure}
\end{frame}

\begin{frame}{\insertsubsectionhead}
  \begin{figure}
    \includegraphics[height=0.7\textheight]{x509-hier.eps}
    \caption{The X.509 certificate hierarchy.
      Image: \cite{Stallings2013nse}.
    }
  \end{figure}
\end{frame}

%\subsection{Public-key Distribution of Secret Keys}
%
%\begin{frame}{\insertsubsectionhead}
%\end{frame}
%
%\subsection{Pretty Good Privacy}
%
%\begin{frame}{\insertsubsectionhead}
%\end{frame}
%
%\subsection{Public-key Infrastructure (PKI)}
%
%\begin{frame}{\insertsubsectionhead}
%\end{frame}
%

\section{Federated Identity Management}

\subsection{Identity Management}

\begin{frame}{\insertsubsectionhead}
  \begin{figure}
    \includegraphics[height=0.7\textheight]{idmgmt-arch.eps}
    \caption{An overview of a generic identity management system.
      Image: \cite{Stallings2013nse}.
    }
  \end{figure}
\end{frame}

\subsection{Identity Federation}

\begin{frame}{\insertsubsectionhead}
  \begin{figure}
    \includegraphics[height=0.7\textheight]{federated.eps}
    \caption{An overview of federated identity systems.
      Image: \cite{Stallings2013nse}.
    }
  \end{figure}
\end{frame}


%%%%%%%%%%%%%%%%%%%%%%

\begin{frame}[allowframebreaks]{Referenser}
	\small
  \printbibliography
\end{frame}

\end{document}

