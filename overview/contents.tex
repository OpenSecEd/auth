\begin{frame}
  \titlepage{}
\end{frame}


% Since this a solution template for a generic talk, very little can
% be said about how it should be structured. However, the talk length
% of between 15min and 45min and the theme suggest that you stick to
% the following rules:  

% - Exactly two or three sections (other than the summary).
% - At *most* three subsections per section.
% - Talk about 30s to 2min per frame. So there should be between about
%   15 and 30 frames, all told.


\section{Authenticating}

\subsection{Machine--user authentication}

\begin{frame}
  \begin{remark}
    \begin{itemize}
      \item The issue we have solved so far is to design means for the system to 
        identify and authenticate different users.

      \item We have another important problem to solve too, how does the user 
        know it is the system he or she is authenticating him- or herself to?

      \item Thus enters the problem of spoofing, phising, and social engineering 
        \dots
    \end{itemize}
  \end{remark}
\end{frame}

\begin{frame}
  \begin{definition}[Spoofing/Masquerading]
    \begin{itemize}
      \item Attacker masquerades as authorized.
      \item To a system: impersonates authorized user.
      \item To a user: impersonates authorized system/UI.
    \end{itemize}
  \end{definition}
\end{frame}

\begin{frame}
  \begin{definition}[Phishing]
    \begin{itemize}
      \item A masquerading attack trying to collect sensitive data.
      \item E.g.\ email from IT department requesting the password.
    \end{itemize}
  \end{definition}

  \pause{}

  \begin{definition}[Social Engineering]
    \begin{itemize}
      \item The general class of attacks on humans.
      \item Exploits fallacies in human psychology.
      \item Parent category of phishing.
      \item Can be very advanced.
    \end{itemize}
  \end{definition}
\end{frame}

\begin{frame}
  \begin{exercise}
    How can we prevent spoofed interfaces?
  \end{exercise}
\end{frame}

\begin{frame}
  \begin{example}
    \begin{itemize}
      \item Show the user the number of failed login attempts.
      \item Show the time and location for the last successful login.
      \item This allows for \emph{detection}.
    \end{itemize}
  \end{example}

  \begin{example}
    \begin{itemize}
      \item We also have the trusted path.
      \item E.g.\ Windows uses the Ctrl+Alt+Del to bring up the authentication 
        dialogue upon login.
      \item This allows for \emph{prevention}.
    \end{itemize}
  \end{example}
\end{frame}

\begin{frame}
  \begin{remark}[Problem with social engineering]
    \begin{itemize}
      \item These are attacks on higher levels, e.g.\ an email or phone call.
      \item Difficult to check algorithmically.
    \end{itemize}
  \end{remark}

  \begin{example}
    \begin{itemize}
      \item Phone call to helpdesk from a \enquote{user} in need.
      \item Stressful situation, willingness to help, \dots
    \end{itemize}
  \end{example}

  \pause{}
  
  \begin{example}[Solution?]
    \begin{itemize}
      \item Authenticated phone calls.
      \item E.g.\ display caller ID clearly.
      \item \enquote{My phone is out of battery, I borrowed a student's}.
    \end{itemize}
  \end{example}
\end{frame}

\begin{frame}
  \begin{solution}
    \begin{itemize}
      \item Educate and train users to spot these attempts.

      \item Keep strong policies for recovering from authentication failures. 

      \item Technological tools and good practices can support users.
    \end{itemize}
  \end{solution}
\end{frame}


%%%%%%%%%%%%%%%%%%%%%%


\begin{frame}[allowframebreaks]
	\small
  \printbibliography{}
\end{frame}

