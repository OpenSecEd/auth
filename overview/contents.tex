\begin{frame}
  \titlepage{}
\end{frame}


% Since this a solution template for a generic talk, very little can
% be said about how it should be structured. However, the talk length
% of between 15min and 45min and the theme suggest that you stick to
% the following rules:  

% - Exactly two or three sections (other than the summary).
% - At *most* three subsections per section.
% - Talk about 30s to 2min per frame. So there should be between about
%   15 and 30 frames, all told.


\section{Introduction}

\subsection{Identification and Authentication}

\begin{frame}
  \begin{itemize}
    \item Authentication is the process of verifying the identity claimed by 
      some system entity.

      \pause{}

      \begin{description}
        \item[Identification] First you enter your username to \emph{identify} 
          yourself.

          \pause{}

        \item[Authentication] Then you enter your password to 
          \emph{authenticate} that you are truly you.
      \end{description}

  \end{itemize}
\end{frame}

\begin{frame}
  \begin{example}[Methods for identification]
    \begin{itemize}
      \item Username or User ID
      \item The person who opened the account
      \item Personal Identification Number (Swe.\ personnummer, Eng.\ Social 
        Security Number)

        \pause{}

      \item Fingerprint
      \item Iris scan
      \item DNA sequence \dots
    \end{itemize}
  \end{example}

  \pause{}

  \begin{exercise}
    Any other methods of identification that you have encountered?
  \end{exercise}
\end{frame}

\begin{frame}
  \begin{example}[Methods for authentication]
    \begin{itemize}
      \item Something you \emph{know}

        \pause{}

      \item Something you \emph{have}

        \pause{}

      \item \emph{Where} you are

        \pause{}

      \item \emph{Who} you are
      \item What you \emph{do}
    \end{itemize}
  \end{example}
\end{frame}

\subsection{Multi-factor authentication}

\begin{frame}
  \begin{example}[Single-factor authentication]
    \begin{description}
      \item[Identification] Username or similar
      \item[Authentication] Something you know, i.e.\ a password
    \end{description}
  \end{example}

  \pause{}

  \begin{example}[Multi-factor authentication]
    \begin{description}
      \item[Identification] Username or similar
      \item[Authentication] Something you know together with something you 
        have, e.g.\ password and mobile phone
    \end{description}
  \end{example}
\end{frame}

\begin{frame}
  \begin{exercise}
    What other authentication methods have you encountered?
  \end{exercise}
\end{frame}


\section[Bootstrapping]{Bootstrapping Authentication}

\subsection{What is bootstrapping?}

\begin{frame}
  \begin{block}{Bootstrapping: A hen-and-egg problem}
    \begin{itemize}
      \item Alice is not registered in our authentication system.
      \item We want to register her as a user in our system.

        \pause{}

      \item How do we know Alice is actually Alice?
      \item Since she's not registered we cannot authenticate her.
    \end{itemize}
  \end{block}

  \pause{}

  \begin{exercise}
    \begin{itemize}
      \item Any quick workarounds that comes to mind?
      \item When is this a problem and when is it not?
    \end{itemize}
  \end{exercise}
\end{frame}

\begin{frame}
  \begin{solution}[We don't care who Alice is]
    \begin{itemize}
      \item We simply set up authentication when Alice creates the account.
      \item Now we can authenticate whoever set up the account.
    \end{itemize}
  \end{solution}

  \pause{}

  \begin{example}
    This is the solution used by most web services.
  \end{example}
\end{frame}

\begin{frame}
  \begin{solution}[We care who Alice actually is]
    \begin{itemize}
      \item We can require ID checks etc.\ to set up the authentication 
        mechanisms using a helpdesk.

        \pause{}

      \item If we have address etc., then we can send the credentials via mail 
        (be it snailmail or email).

    \end{itemize}
  \end{solution}

  \pause{}

  \begin{example}[The university account]
    This is how your university account was set up.
  \end{example}
\end{frame}

\begin{frame}
  \begin{example}[Signal, WhatsApp, \dots]
    \begin{itemize}
      \item The identity is a (mobile) phone number.
      \item Send a text message with a code.
    \end{itemize}
  \end{example}

  \pause{}

  \begin{remark}
    \begin{itemize}
      \item Phone provider can impersonate.
      \item Government can impersonate (forcing phone provider).
    \end{itemize}
  \end{remark}
\end{frame}

\subsection{Problems with Bootstrapping}

\begin{frame}
  \begin{enumerate}
    \item Attacker \emph{intercepts} a password on account creation.
      \begin{itemize}
        \item User starts bootstrapping.
        \item Password is sent to user.
        \item Attacker grabs password.
      \end{itemize}

      \pause{}

    \item Attacker \emph{impersonates} the legitimate user.
      \begin{itemize}
        \item Attacker starts bootstrapping.
        \item User remains unaware.
        \item Service cannot distinguish attacker and user.
      \end{itemize}
  \end{enumerate}
\end{frame}

\begin{frame}
  \begin{itemize}
    \item It can be costly to manage.

    \item Sometimes it is a continuous process, if the same bootstrapping 
      procedure is also used for \emph{recovery from failure}.

    \item Make sure the system can handle forgotten, lost or aged 
      authentication means.

  \end{itemize}
\end{frame}

\subsection{Single Sign-On}

\begin{frame}
  \begin{itemize}
    \item We could let someone else who has solved the problem already do the 
      authentication for us.

    \item This way the user only needs one username and password, and he or she 
      only needs to sign in once.

    \item However, this makes the SSO provider a very attractive target.

    \item And they are forced to solve our problem anyway.

    \item The problem is, now we need to trust them to do it properly \dots
  \end{itemize}
\end{frame}

\begin{frame}
  \begin{example}[We don't care who Alice is]
    We can use Google, Facebook etc.
  \end{example}

  \begin{example}[We care who Alice is]
    We can use e.g.\ BankID.
  \end{example}

  \begin{remark}
    The SSO-service must have done bootstrapping as rigorously as we would 
    have.
  \end{remark}
\end{frame}


\section{Authenticating}

\subsection{Time of Check to Time of Use}

\begin{frame}
  \begin{exercise}
    \begin{itemize}
      \item Whenever we authenticate a user, we do this for a purpose.

      \item When does this authentication take place in relation to when we make 
        use of it?
    \end{itemize}
  \end{exercise}
\end{frame}

\begin{frame}
  \begin{example}
    \begin{itemize}
      \item Usually we authenticate a user in the beginning of a session, e.g.\ 
        at login.

      \item Equally often we assume the user is authenticated during the entire 
        session, even when fetching coffee, going by the printer -- or even when 
        out to lunch.

      \item Who knows what happens when the user is away from the computer, one 
        thing is for sure: the computer will not know the difference!

    \end{itemize}
  \end{example}
\end{frame}

\begin{frame}
  \begin{solution}
    \begin{itemize}
      \item This problem can be solved with \emph{repeated authentication}.

      \item We could lock our system, either manually or by time-out.

      \item We could also authenticate anew when we need to do something 
        requiring more privileges, and if it has been a while since last time -- 
        compare with sudo(8).
    \end{itemize}
  \end{solution}
\end{frame}

\subsection{Phishing, Spoofing, Social Engineering}

\begin{frame}
  \begin{remark}
    \begin{itemize}
      \item The issue we have solved so far is to design means for the system to 
        identify and authenticate different users.

      \item We have another important problem to solve too, how does the user 
        know it is the system he or she is authenticating him- or herself to?

      \item Thus enters the problem of spoofing, phising, and social engineering 
        \dots
    \end{itemize}
  \end{remark}
\end{frame}

\begin{frame}
  \begin{definition}[Spoofing/Masquerading]
    \begin{itemize}
      \item Attacker masquerades as authorized.
      \item To a system: impersonates authorized user.
      \item To a user: impersonates authorized system/UI.
    \end{itemize}
  \end{definition}
\end{frame}

\begin{frame}
  \begin{definition}[Phishing]
    \begin{itemize}
      \item A masquerading attack trying to collect sensitive data.
      \item E.g.\ email from IT department requesting the password.
    \end{itemize}
  \end{definition}

  \pause{}

  \begin{definition}[Social Engineering]
    \begin{itemize}
      \item The general class of attacks on humans.
      \item Exploits fallacies in human psychology.
      \item Parent category of phishing.
      \item Can be very advanced.
    \end{itemize}
  \end{definition}
\end{frame}

\begin{frame}
  \begin{exercise}
    How can we prevent spoofed interfaces?
  \end{exercise}
\end{frame}

\begin{frame}
  \begin{example}
    \begin{itemize}
      \item Show the user the number of failed login attempts.
      \item Show the time and location for the last successful login.
      \item This allows for \emph{detection}.
    \end{itemize}
  \end{example}

  \begin{example}
    \begin{itemize}
      \item We also have the trusted path.
      \item E.g.\ Windows uses the Ctrl+Alt+Del to bring up the authentication 
        dialogue upon login.
      \item This allows for \emph{prevention}.
    \end{itemize}
  \end{example}
\end{frame}

\begin{frame}
  \begin{remark}
    \begin{itemize}
      \item We authentication of the system to the user.
      \item In some sense, it boils down to trust.
    \end{itemize}
  \end{remark}
\end{frame}

\begin{frame}
  \begin{remark}[Problem with social engineering]
    \begin{itemize}
      \item These are attacks on higher levels.
      \item E.g.\ an email or phone call.
      \item Difficult to check algorithmically.
    \end{itemize}
  \end{remark}

  \begin{example}
    \begin{itemize}
      \item Phone call to helpdesk from a \enquote{user} in need.
      \item Stressful situation, willingness to help, \dots
    \end{itemize}
  \end{example}
\end{frame}

\begin{frame}
  \begin{example}[Solution?]
    \begin{itemize}
      \item Authenticated phone calls.
      \item E.g.\ display caller ID clearly.
      \item \enquote{My phone is out of battery, I borrowed a student's}.
    \end{itemize}
  \end{example}

  \begin{solution}
    \begin{itemize}
      \item Educate and train users to spot these attempts.

      \item Keep strong policies for recovering from authentication failures. 

      \item Technological tools and good practices can support users.
    \end{itemize}
  \end{solution}
\end{frame}

% XXX Continue reworking material below

\subsection{Guessing Passwords}

\begin{frame}
  \begin{itemize}
    \item Guessing passwords is like searching for a needle in a haystack.

      \pause{}

    \item (Un)fortunately, the needle is placed by a human --- not uniformly 
      randomly!
    \item This makes guessing easier.
    \item Human-chosen passwords will only occupy parts of the password space.
  \end{itemize}
\end{frame}

\begin{frame}
  \begin{itemize}
    \item The effort is a spectrum.

    \item It ranges from brute-force exhaustive search \dots

      \pause{}

    \item \dots via \enquote{educated guessing} \dots

      \pause{}

    \item \dots to getting the password from the user directly.
  \end{itemize}
\end{frame}

\begin{frame}
  \begin{example}[Basic guessing]
    \begin{itemize}
      \item Using dictionaries of words.
      \item Adapt to guesses to password policy, if known.
      \item \dots
    \end{itemize}
  \end{example}

  \pause{}

  \begin{example}[Improved guessing]
    Take grammar into account, depending on the password 
    type~\cite{Bonneau2012ghs,Bonneau2012lpo}.
  \end{example}
\end{frame}

\begin{frame}
  \begin{example}[Learn from humans]
    \begin{itemize}
      \item Use machine learning~\cite{JohnTheRipper,OMEN,WeirPCFG}.
      \item Train algorithm on leaked password databases.
      \item Generate list of password-looking guesses.
    \end{itemize}
  \end{example}
\end{frame}

\begin{frame}
  \begin{remark}
    \begin{itemize}
      \item This is relevant when the user has chosen a password.
      \item In the majority of situations it's not.
    \end{itemize}
  \end{remark}

  \pause{}

  \begin{example}
    \begin{itemize}
      \item There are many devices with default passwords.
      \item E.g.\ home routers, \dots
    \end{itemize}
  \end{example}
\end{frame}

\begin{frame}
  \begin{itemize}
    \item Change default passwords, those are the obvious first guess.

      \pause{}

    \item Increase the length of the passwords~\cite{Komanduri2011opa}.

    \item We could increase complexity, we know this is popular --- 
      but~\cite{Komanduri2011opa} shows it is worse for users than further 
      increasing the length.
  \end{itemize}
\end{frame}

\begin{frame}
  \begin{itemize}
    \item Generate passwords for users: this will likely reduce security by use 
      of post-it notes.

      \pause{}

    \item Password ageing: annoying with too short intervals.
    \item Will reduce security once users introduce systems to remember their 
      last changed password.

      \pause{}

    \item Remove online guessing by limited login attempts: at the cost of 
      possible denial of service.

  \end{itemize}
\end{frame}


\section[Securing]{Securing Authentication}

\subsection{The Password File}

\begin{frame}
  \begin{itemize}
    \item Once we have data which can be used to authenticate users, we need to 
      store this data somewhere for future authentication.

    \item Traditionally, in the case of passwords, there has always been a 
      password file (or database) containing all users' passwords.

    \item Naturally we have to protect this data, otherwise if someone got hold 
      of it he or she could impersonate any user in the system.

  \end{itemize}
\end{frame}

\begin{frame}
  \begin{itemize}
    \item There are of course different approaches to protect it.

    \item One way is to encrypt it.

    \item Another is to let the operating system's or database's access control 
      protect it.

    \item A third way is to do both.

  \end{itemize}
\end{frame}

\begin{frame}
  \begin{itemize}
    \item To encrypt the file, we do not actually need to encrypt it.

    \item We could apply an irreversible transformation on all the passwords.

    \item Using a one-way function, or cryptographic hash function, we can make 
      the passwords unreadable but still verifiable.

    \item Let \(h\) be a hash function and \(p\) a password, then we simply 
      compute \(h(p)\) to store in our database.

    \item When the user wants to authenticate with a password \(p^\prime\) we 
      simply compute \(h(p^\prime)\) and compare to the stored \(h(p)\).

  \end{itemize}
\end{frame}

\begin{frame}
  \begin{itemize}
    \item The function \(h\) is selected to be a very slow one-way function, 
      maybe we iterate its application to the password (like 10\,000 to 
      100\,000 times, not only 5).

    \item This way we can slow down guessing attacks should anyone ever come 
      across our password database.

    \item However, as our password file is currently structured, we can still 
      see if two users have the same password -- they would have the same hash 
      value.

    \item This way we can guess the password for all users at once: we make 
      a guess, check if it matches \emph{any} user's password.

    \item To counter this we add a \emph{salt}, this is a small value (e.g.\ 
      128 bits) selected randomly for each user.

    \item This way all password hashes will be unique and the attack would have 
      to be on a per-password basis.

  \end{itemize}
\end{frame}

\begin{frame}
  \begin{itemize}
    \item bcrypt implements all this functionality.
    \item It should also be available in most languages and libraries.
  \end{itemize}
\end{frame}

\subsection{Alternative Approaches}

\begin{frame}
  \begin{itemize}
    \item We mentioned in the beginning some alternatives to passwords, these 
      were:
      \begin{itemize}
        \item who you are,
        \item where you are,
        \item what you do,
        \item something you know, and
        \item something you have.
      \end{itemize}

    \item Using two or more of these together gave us two- or multiple-factor 
      authentication, this reduces the risk of false positives.

  \end{itemize}
\end{frame}

\begin{frame}
  \begin{itemize}
    \item ``Who you are'' is becoming more popular as a basis for 
      authentication.

    \item This covers biometrics like fingerprints or iris scans.

    \item ``What you do'' is also a biometric, however, the focus of this one 
      is on things you do, e.g.\ how you write your signature -- including the 
      resulting signature, the writing speed, the pressure at different points, 
      etc.

    \item Finally, ``where you are'' is also becoming easier to apply with 
      positioning systems in smartphones and IP addresses of computers.

    \item E.g.\ Google employs where you are if you have enabled two-factor 
      authentication, then you will have too verify with your phone if your IP 
      address changes.
  \end{itemize}
\end{frame}

\begin{frame}
  \begin{itemize}
    \item You can also take this to the new level.

      % XXX add ref to IEEE Security and Privacy article on web fingerprinting
      % XXX and tracking
    \item Some web sites use visitor fingerprinting techniques to identify 
      users.

    \item This can be used to detect if the real user is authenticating or not.

    \item However, these fingerprinting mechanisms are typically used for other 
      purposes, to track users' browsing habits to serve targeted ads to them.

    \item EFF developed a tool to illustrate the privacy invasion of browser 
      fingerprinting in~\cite{Eckersley2010hui}.
      \begin{itemize}
        \item You can test browser fingerprinting at URL 
          \url{https://panopticlick.eff.org/}.
      \end{itemize}

  \end{itemize}
\end{frame}

\begin{frame}
  \begin{itemize}
    \item A problem with all of these is the accuracy and failure rate of the 
      technologies needed to sample the data used in authentication.

    \item E.g.\ a fingerprint reader is prone to error due to greasy fingers, 
      complex passwords can be typed incorrectly, etc.

    \item And since it introduces another system in some cases, it also 
      introduces trust issues for the user -- is this an authentic fingerprint 
      reader, is it really the Google sign-in in this IFRAME, or is it one that 
      will steal my credentials?

  \end{itemize}
\end{frame}

% XXX include failure rates from Gollmann

\subsection{Anonymous Credentials}

\begin{frame}
  \begin{example}[Age limits]
    \begin{itemize}
      \item Bob wants to go see a film in cinema.
      \item Bob looks very young so Alice who works there wants to have proof 
        of his age.

        \pause{}

      \item Show physical ID, reveals name, exact date of birth, \dots
    \end{itemize}
  \end{example}

  \pause{}

  \begin{exercise}
    \begin{itemize}
      \item That's a bit overkill, right?
      \item What does Alice actually need to know?
      \item In what direction must we move to achieve that?
    \end{itemize}
  \end{exercise}
\end{frame}

\begin{frame}
  \begin{block}{What Alice needs?}
    She must be convinced that Bob is older than 15.
  \end{block}

  \pause{}

  \begin{alertblock}{How can she learn that?}
    \begin{enumerate}
      \item She has known Bob since he was born, so she knows.

        \pause{}

      \item She can ask someone \emph{she trusts} who knows Bob is older than 
        15.
    \end{enumerate}
  \end{alertblock}
\end{frame}

\begin{frame}
  \begin{alertblock}{But how can she do that?}
    \begin{enumerate}
      \item The trusted person who knows Bob is with Alice.

        \pause{}

      \item Alice can send a picture to the other person who verifies.
        \begin{itemize}
          \item This requires an \emph{authenticated} channel.
        \end{itemize}

        \pause{}

      \item The trusted person made a certificate for Bob showing that he's 
        older than 15.
        \begin{itemize}
          \item Alice must be able to \emph{verify} the certificate.
          \item Bob must bring this certificate with himself everywhere.
        \end{itemize}

    \end{enumerate}
  \end{alertblock}
\end{frame}

\begin{frame}
  \begin{alertblock}{Alice interacts with the trusted person}
    \begin{itemize}
      \item Gaah, but Bob doesn't want the trusted person (his parents) to know 
        he's at the cinema right now!
      \item It's a small cinema so they'll know which film he sees if they 
        learn when he's there.
    \end{itemize}
  \end{alertblock}

  \pause{}

  \begin{alertblock}{Alice reads and verifies the certificate}
    \begin{itemize}
      \item Phew, she accepted the note from his parents.
      \item But now Alice learned all those embarrassing things in 
        there.
        \begin{itemize}
          \item And Bob who has a crush on Alice \dots
        \end{itemize}
    \end{itemize}
  \end{alertblock}
\end{frame}

\begin{frame}
  \begin{block}{The idea}
    \begin{itemize}
      \item What if Bob could convince Alice
        \begin{itemize}
          \item that he has a certificate saying he's older than 15,
          \item and is signed by someone Alice trusts.
        \end{itemize}
      \item Wouldn't that be awesome?
    \end{itemize}
  \end{block}
\end{frame}

\begin{frame}
  \begin{example}[Anonymous 
    Credentials\footfullcite{ElectronicIdentitiesNeedPrivateCredentials}]
    \begin{itemize}
      \item Makes heavy use of zero-knowledge proofs of knowledge.
      \item Can prove equalities, inequalities, knowledge, ownership, \dots
      \item Implementations and approaches:
        \begin{description}
          \item[Identity Mixer]
            \url{https://www.research.ibm.com/labs/zurich/idemix/}
          \item[U-Prove]
            \url{http://research.microsoft.com/en-us/projects/u-prove/}
          \item[AnonPass]
            \url{https://eprint.iacr.org/2013/317}
          \item[IRMA]
            \url{https://www.irmacard.org/irma/}
        \end{description}
    \end{itemize}
  \end{example}
\end{frame}


%%%%%%%%%%%%%%%%%%%%%%


\begin{frame}[allowframebreaks]
	\small
  \printbibliography{}
\end{frame}

