%\documentclass[handout]{beamer}
\documentclass{beamer}
\usepackage[utf8]{inputenc}
\usepackage[T1]{fontenc}
\usepackage[swedish,english]{babel}
\usepackage{url}
\usepackage{graphicx}
\usepackage{color}
\usepackage{subfig}
\usepackage{multicol}
\usepackage{amssymb,amsmath,amsthm}
\usepackage{booktabs}
\usepackage[binary,squaren]{SIunits}
\usepackage{listings}

\setbeamertemplate{bibliography item}[text]
\usepackage[natbib,style=alphabetic,maxbibnames=99]{biblatex}
\addbibresource{overview.bib}

\mode<presentation>{%
  \usetheme{Frankfurt}
  \setbeamercovered{transparent}
  \usecolortheme{seagull}
}
\setbeamertemplate{footline}{\insertframenumber}

\title[Authentication]{%
  Identification and Authentication
}
\author{Daniel Bosk\footnote{%
  This work is licensed under the Creative Commons Attribution-ShareAlike 3.0 
  Unported license.
	To view a copy of this license, visit 
	\url{http://creativecommons.org/licenses/by-sa/3.0/}.
}}
\institute[MIUN ICS]{%
  Department of Information and Communication Systems (ICS),\\
  Mid Sweden University, Sundsvall.
}
\date{\today}

%\pgfdeclareimage[height=0.65cm]{university-logo}{MU_logotyp_int_CMYK.pdf}
%\logo{\pgfuseimage{university-logo}}

\AtBeginSection[]{%
  \begin{frame}<beamer>{Overview}
    \tableofcontents[currentsection]
  \end{frame}
}
%\AtBeginSubsection[]{%
%  \begin{frame}<beamer>
%    \begin{center}
%      \insertsectionhead
%    \end{center}
%  \end{frame}
%}

\begin{document}

\begin{frame}
  \titlepage{}
\end{frame}

\begin{frame}{Overview}
  \tableofcontents
  % You might wish to add the option [pausesections]
\end{frame}


% Since this a solution template for a generic talk, very little can
% be said about how it should be structured. However, the talk length
% of between 15min and 45min and the theme suggest that you stick to
% the following rules:  

% - Exactly two or three sections (other than the summary).
% - At *most* three subsections per section.
% - Talk about 30s to 2min per frame. So there should be between about
%   15 and 30 frames, all told.


\section[Bootstrapping]{Bootstrapping Authentication}

\subsection{What Is Authentication?}

\begin{frame}{\insertsubsectionhead}
  \begin{itemize}
    \item Authentication is the process of verifying the identity claimed by 
      some system entity.

    \item I.e.\ first you enter your username to \emph{identify} yourself.

    \item Then you enter your password to \emph{authenticate} that you are 
      truly you.

  \end{itemize}
\end{frame}

\subsection{Types of Identification and Authentication}

\begin{frame}{\insertsubsectionhead}{Identification}
  \begin{itemize}
    \item Username or User ID
    \item The person who opened the account
    \item Personal Identification Number (Swe.\ personnummer, Eng.\ Social 
      Security Number)
    \item Fingerprint
    \item Iris scan
    \item DNA sequence \dots
    \item Cryptographic key or certificate
  \end{itemize}
\end{frame}

\begin{frame}{\insertsubsectionhead}{Authentication}
  \begin{itemize}
    \item \emph{Who} you are
    \item \emph{Where} you are
    \item What you \emph{do}
    \item Something you \emph{have}
    \item Something you \emph{know}
  \end{itemize}
\end{frame}

\begin{frame}{\insertsubsectionhead}
  \begin{itemize}
    \item The most common way of authenticating is by something you know, i.e.\ 
      passwords.

    \item There is also the possibility of combining with the others to have 
      two-factor authentication.

    \item E.g., it is common to have something you know together with something 
      you have, e.g.\ password and mobile phone.

  \end{itemize}
\end{frame}

\subsection{Bootstrapping Authentication}

\begin{frame}{\insertsubsectionhead}
  \begin{itemize}
    \item All systems have begun sometime and somewhere, and they always will.

    \item Whenever a user is added to an authentication system we need 
      a beginning for that user.

    \item How do we know it is the correct user before he or she can 
      authenticate with our system?

    \item It is kind of a ``the hen or the egg'' problem.

  \end{itemize}
\end{frame}

\begin{frame}{\insertsubsectionhead}
  \begin{itemize}
    \item Bootstrapping authentication is about getting authentication going.

    \item To do this we can have several approaches.

    \item First, if we do not care about who the person is, we can just 
      bootstrap authentication by that person setting up authentication 
      mechanisms at account registration.

    \item This is very common, usually Web services only want to authenticate 
      the person who registered the account.

    \item If we care a bit more, then we can require ID checks etc.\ to set up 
      the authentication mechanisms using a helpdesk.

    \item If we have address etc.\, then we can send the things the user needs 
      to sign in via mail (be it snailmail or email).

    \item Depending on what we base authentication, the set up of the 
      mechanisms can be to register a password, a phone number for texting 
      verification codes, or a fingerprint scan, and so on.
  \end{itemize}
\end{frame}

\subsection{Problems with Bootstrapping}

\begin{frame}{\insertsubsectionhead}
  \begin{itemize}
    \item Attacker \emph{intercepts} a password on account creation.
    \item Attacker \emph{impersonates} the legitimate user.
  \end{itemize}
\end{frame}

\begin{frame}{\insertsubsectionhead}
  \begin{itemize}
    \item It can be costly to manage.

    \item Sometimes it is a continuous process, if the same bootstrapping 
      procedure is also used for \emph{recovery from failure}.

    \item Make sure your system can handle forgotten, lost, or aged 
      authentication means.

  \end{itemize}
\end{frame}

\subsection{Single Sign-On}

\begin{frame}{\insertsubsectionhead}
  \begin{itemize}
    \item We could let someone else who has solved the problem already do the 
      authentication for us, e.g.\ Google or Facebook.

    \item This way the user only needs one username and password, and he or she 
      only needs to sign in once.

    \item However, this makes the SSO provider a very attractive target.

    \item And they are forced to solve our problem anyway.

    \item The problem is, now we need to trust them to do it properly \dots
  \end{itemize}
\end{frame}

\begin{frame}{\insertsubsectionhead}
  \begin{itemize}
    \item Usually these services doesn't provide reliable identities, just that 
      it's the user who registered the account.

    \item If we need real identities, then we'd need to use BankID or similar 
      solution.
  \end{itemize}
\end{frame}


\section{Authenticating}

\subsection{Time of Check to Time of Use}

\begin{frame}{\insertsubsectionhead}
  \begin{itemize}
    \item Whenever we authenticate a user, we do this for a purpose.

    \item When does this authentication take place in relation to when we make 
      use of it?

    \item Usually we authenticate a user in the beginning of a session, e.g.\ 
      at login.

    \item Equally often we assume the user is authenticated during the entire 
      session, even when fetching coffee, going by the printer -- or even when 
      out to lunch.

    \item Who knows what happens when the user is away from the computer, one 
      thing is for sure: the computer will not know the difference!

  \end{itemize}
\end{frame}

\begin{frame}{\insertsubsectionhead}
  \begin{itemize}
    \item This problem can be solved with \emph{repeated authentication}.

    \item We could lock our system, either manually or by timeout.

    \item We could also authenticate anew when we need to do something 
      requiring more privileges, and if it has been a while since last time -- 
      compare with sudo(8).
  \end{itemize}
\end{frame}

\subsection{Phising, Spoofing, Social Engineering}

\begin{frame}{\insertsubsectionhead}
  \begin{itemize}
    \item The issue we have solved so far is to design means for the system to 
      identify and authenticate different users.

    \item We have another important problem to solve too, how does the user 
      know it is the system he or she is authenticating him- or herself to?

    \item Thus enters the problem of spoofing, phising, and social engineering 
      \dots
  \end{itemize}
\end{frame}

\begin{frame}{\insertsubsectionhead}
  \begin{description}
    \item[Spoofing] A spoofing attack presents the user with a fake interface, 
      tricking the user into believing he or she is communicating with the 
      correct system.

    \item[Phishing] A phishing attack asks users for their password, or other 
      information, under false pretences, e.g.\ that the IT department is 
      changing the security system and needs the user to reenter the password.

    \item[Social Engineering] The technique of social engineering is to exploit 
      fallacies in human psychology.
      Here, the attacker might call the user and trick said user into doing 
      what the attacker wants.
      Or, the attacker might call helpdesk, impersonating a user and require 
      a password reset.
  \end{description}
\end{frame}

\begin{frame}{\insertsubsectionhead}
  \begin{itemize}
    \item To counter spoofing one could show the user the number of failed 
      login attempts, the time and location for the last successful login, etc.

    \item We also have the trusted path, e.g.\ Windows uses the Ctrl+Alt+Del to 
      bring up the authentication dialogue upon login.

    \item We could also have some other type of authentication of the system to 
      the user.

    % XXX make independent of secure protocols?
    \item We will return to this subject when we discuss secure protocols.

  \end{itemize}
\end{frame}

\begin{frame}{\insertsubsectionhead}
  \begin{itemize}
    \item Countering phishing and social engineering is a bit harder, and 
      requires a different approach.

    \item The most obvious approach is to educate and train users to spot these 
      attempts.

    \item However, since these techniques exploit weaknesses in human nature, 
      it can be hard.

    \item But keeping strong policies for recovering from authentication 
      failures, e.g.\ forgotten passwords, can mitigate social engineering 
      attempts.

    \item Some technological tools and good practices can support users in 
      avoiding phishing attempts too.
  \end{itemize}
\end{frame}

\subsection{Guessing Passwords}

\begin{frame}{\insertsubsectionhead}
  \begin{itemize}
    \item There are generally two approaches to guessing passwords.

    \item The first one is exhaustive search (brute force), which is to test 
      all possible combinations of valid characters.

    \item The second one is educated search, here we use some more knowledge.
  \end{itemize}
\end{frame}

\begin{frame}{\insertsubsectionhead}
  \begin{itemize}
    \item Possible educated search is using dictionaries, adapt to password 
      policy, etc.

    \item Some more advanced techniques for guessing passwords have also been 
      developed.

    \item E.g.\ one could take grammar into account, depending on the password 
      type~\cite{Bonneau2012ghs,Bonneau2012lpo}.
  \end{itemize}
\end{frame}

\begin{frame}{\insertsubsectionhead}
  \begin{itemize}
    \item There are also some measures to be taken to counter password 
      guessing.

    \item First, change default passwords, those are the obvious first guess.

    \item Increase the length of the passwords.

    \item We could increase complexity, we know this is popular --- 
      but~\cite{Komanduri2011opa} shows it is worse for users than further 
      increasing the length.
  \end{itemize}
\end{frame}

\begin{frame}{\insertsubsectionhead}
  \begin{itemize}
    \item We can also generate passwords for users, although this might reduce 
      security by use of post-it notes.

    \item We can also introduce password ageing, can be annoying with too short 
      intervals -- and will reduce security once users introduce systems to 
      remember their last changed password.

    \item Finally we can remove online guessing by introducing limited login 
      attempts, however at the cost of possible denial of service.

  \end{itemize}
\end{frame}


\section[Securing]{Securing Authentication}

\subsection{The Password File}

\begin{frame}{\insertsubsectionhead}
  \begin{itemize}
    \item Once we have data which can be used to authenticate users, we need to 
      store this data somewhere for future authentication.

    \item Traditionally, in the case of passwords, there has always been a 
      password file (or database) containing all users' passwords.

    \item Naturally we have to protect this data, otherwise if someone got hold 
      of it he or she could impersonate any user in the system.

  \end{itemize}
\end{frame}

\begin{frame}{\insertsubsectionhead}
  \begin{itemize}
    \item There are of course different approaches to protect it.

    \item One way is to encrypt it.

    \item Another is to let the operating system's or database's access control 
      protect it.

    \item A third way is to do both.

  \end{itemize}
\end{frame}

\begin{frame}{\insertsubsectionhead}
  \begin{itemize}
    \item To encrypt the file, we do not actually need to encrypt it.

    \item We could apply an unreversible transformation on all the passwords.

    \item Using a one-way function, or cryptographic hash function, we can make 
      the passwords unreadable but still verifiable.

    \item Let \(h\) be a hash function and \(p\) a password, then we simply 
      compute \(h(p)\) to store in our database.

    \item When the user wants to authenticate with a password \(p^\prime\) we 
      simply compute \(h(p^\prime)\) and compare to the stored \(h(p)\).

  \end{itemize}
\end{frame}

\begin{frame}{\insertsubsectionhead}
  \begin{itemize}
    \item The function \(h\) is selected to be a very slow one-way function, 
      maybe we iterate its application to the password (like 10\,000 to 
      100\,000 times, not only 5).

    \item This way we can slow down guessing attacks should anyone ever come 
      across our password database.

    \item However, as our password file is currently structured, we can still 
      see if two users have the same password -- they would have the same hash 
      value.

    \item This way we can guess the password for all users at once: we make 
      a guess, check if it matches \emph{any} user's password.

    \item To counter this we add a \emph{salt}, this is a small value (e.g.\ 
      128 bits) selected randomly for each user.

    \item This way all password hashes will be unique and the attack would have 
      to be on a per-password basis.

  \end{itemize}
\end{frame}

\begin{frame}{\insertsubsectionhead}
  \begin{itemize}
    \item bcrypt implements all this functionality.
    \item It should also be available in most languages and libraries.
  \end{itemize}
\end{frame}

\subsection{Alternative Approaches}

\begin{frame}{\insertsubsectionhead}
  \begin{itemize}
    \item We mentioned in the beginning some alternatives to passwords, these 
      were:
      \begin{itemize}
        \item who you are,
        \item where you are,
        \item what you do,
        \item something you know, and
        \item something you have.
      \end{itemize}

    \item Using two or more of these together gave us two- or multiple-factor 
      authentication, this reduces the risk of false positives.

  \end{itemize}
\end{frame}

\begin{frame}{\insertsubsectionhead}
  \begin{itemize}
    \item ``Who you are'' is becoming more popular as a basis for 
      authentication.

    \item This covers biometrics like fingerprints or iris scans.

    \item ``What you do'' is also a biometric, however, the focus of this one 
      is on things you do, e.g.\ how you write your signature -- including the 
      resulting signature, the writing speed, the pressure at different points, 
      etc.

    \item Finally, ``where you are'' is also becoming easier to apply with 
      positioning systems in smartphones and IP addresses of computers.

    \item E.g.\ Google employs where you are if you have enabled two-factor 
      authentication, then you will have too verify with your phone if your IP 
      address changes.
  \end{itemize}
\end{frame}

\begin{frame}{\insertsubsectionhead}
  \begin{itemize}
    \item You can also take this to the new level.

      % XXX add ref to IEEE Security and Privacy article on Web fingerprinting
      % XXX and tracking
    \item Some Web sites use visitor fingerprinting techniques to identify 
      users.

    \item This can be used to detect if the real user is authenticating or not.

    \item However, these fingerprinting mechanisms are typically used for other 
      purposes, to track users' browsing habits to serve targeted ads to them.

    \item EFF developed a tool to illustrate the privacy invasion of browser 
      fingerprinting in~\cite{Eckersley2010hui}.
      \begin{itemize}
        \item You can test browser fingerprinting at URL 
          \url{https://panopticlick.eff.org/}.
      \end{itemize}

  \end{itemize}
\end{frame}

\begin{frame}{\insertsubsectionhead}
  \begin{itemize}
    \item A problem with all of these is the accuracy and failure rate of the 
      technologies needed to sample the data used in authentication.

    \item E.g.\ a fingerprint reader is prone to error due to greasy fingers, 
      complex passwords can be typed incorrectly, etc.

    \item And since it introduces another system in some cases, it also 
      introduces trust issues for the user -- is this an authentic fingerprint 
      reader, is it really the Google sign-in in this IFRAME, or is it one that 
      will steal my credentials?

  \end{itemize}
\end{frame}

% XXX include failure rates from Gollmann


%%%%%%%%%%%%%%%%%%%%%%

\begin{frame}{References}
	\small
  \printbibliography{}
\end{frame}

\end{document}
