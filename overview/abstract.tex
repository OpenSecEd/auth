Authentication is part of the core of security.
An entity claims something, a property or an identity, authentication is about 
verifying or rejecting any such claim.
We will discuss three aspects of authentication:
user-to-machine (and user-to-user),
machine-to-user,
machine-to-machine.
For user authentication we will start with the traditional something you know, 
something you have and something you are and then look beyond.

More specifically, the session should prepare you to be able to
\begin{itemize}
  \item \emph{understand} the authentication and usability problems of 
    authentication involving users.
  \item \emph{analyse} the requirements for authentication in a situation and 
    \emph{design} an authentication system with desired authentication 
    properties and usability.
\end{itemize}

Why we want to do this and how we can accomplish this is treated in Chapter 
4 in~\cite{Gollmann2011cs}.
Anderson also treats this topic \cite[Chap.~2]{Anderson2008sea}, although in 
a wider perspective with less technical details.
When you have studied this material you should do exercises 4.2, 4.3, 4.4 and 
4.6 in~\cite{Gollmann2011cs}.
