\subsection{Lösenord}

\begin{frame}{Användbarhet?}
  \begin{itemize}
    \item Svårt att komma ihåg detaljer som används sällan.
    \item Svårt att komma ihåg detaljer som ändras ofta.
    \item Svårt att komma ihåg och särskilja många liknande detaljer.
    \item Svårt att minnas ord utan betydelse.
    \item Kan ej glömma på begäran.
    \item Att minnas är svårare än att känna igen.
  \end{itemize}
\end{frame}

\begin{frame}
  \begin{itemize}
    \item Enklare att komma ihåg saker som används ofta.
    \item Enklare att minnas saker i kontext.
    \item Men \dots
  \end{itemize}
\end{frame}

\begin{frame}
  \begin{figure}
    \includegraphics[height=0.65\textheight]{password_strength.png}
    \caption{%
      Hard to remember, easy to guess.
      Easy to remember, hard to guess.
      Bild:~\cite{xkcd936}.
    }
  \end{figure}
\end{frame}

\subsection{Alternatives}

\begin{frame}
  \begin{block}{Simson Garfinkel:}
    \begin{itemize}
      \item Something you had once
      \item Something you've forgotten
      \item Something you once were.
    \end{itemize}
  \end{block}
\end{frame}

\begin{frame}
  \begin{block}{Really?}
    \begin{description}
      \item[Vet] Lösenord.
      \item[Har] Koddosa, som oftast skyddas av ett lösenord.
      \item[Är] Fingeravtryck, som oftast kombineras med ett lösenord.
    \end{description}
  \end{block}
\end{frame}

\begin{frame}{Komplexiteten hos lösenord}
  \begin{itemize}
    \item PIN-koden för betalkortet, har endast tre försök sedan slutar kortet 
      att fungera.

      \pause{}

    \item Lösenordet för webbmailen, vore väldigt jobbigt om den blev låst.
      Hur låsa upp?

      \pause{}

    \item Krypterat data, har ej kontroll över antal försök.

  \end{itemize}
\end{frame}

\subsection{Workarounds}

\begin{frame}
  \begin{example}[Other types of passwords]
    \begin{itemize}
      \item Personnummer (även användarnamn).
      \item Kortnummer, medlemsnummer.
      \item Husdjurets namn.
      \item \enquote{Mother's maiden name}.
    \end{itemize}
  \end{example}
\end{frame}

\begin{frame}
  \begin{figure}
    \includegraphics[height=0.65\textheight]{pet_security_question.png}
    \caption{En seriestrip som antyder det bisarra med säkerhetsfrågor.
    Namnge dina husdjur med omsorg, du kommer att använda deras namn som 
    säkerhetsfråga resten av livet.}
  \end{figure}
\end{frame}

\subsection{Problems to solve}

\begin{frame}
  \begin{enumerate}
    \item Kommer användaren att mata in rätt lösenord tillräckligt ofta?

    \item Kan användaren minnas lösenordet, eller kommer denne att skriva ner 
      det på en lapp?
      Väljer användaren ett lösenord som är lätt att gissa?

  \end{enumerate}
\end{frame}

\begin{frame}
  \begin{example}[Entering passwords]
    \begin{itemize}
      \item Muntligen ange ett nummer: hotel-, biljettbokningar, hämta ut paket 
        från Posten.

      \item Mata in långa sifferkombinationer: mjukvarulicenser, refillkort, 
        OCR-nummer för räkningar.

      \item Att skriva dem i grupper om tre till fyra underlättar avsevärt.

      \item Längre lösenord, större sannolikhet att skriva fel.

    \end{itemize}
  \end{example}
\end{frame}

\begin{frame}
  \begin{example}[Remembering passwords]
    \begin{itemize}
      \item Välj ett lösenord du inte kan minnas och skriv inte ner det.

      \item xkcd:s \enquote{correct horse battery staple}, enkelt att komma 
        ihåg men svårare att skriva.

      \item Men om man bara behöver skriva det sällan, då är det mindre problem.

    \end{itemize}
  \end{example}
\end{frame}

\begin{frame}
  \begin{itemize}
    \item \citet{Komanduri2011opa} gjorde en undersökning om säkerhet och 
      användbarhet hos olika lösenordspolicyer.

      \pause{}

    \item Hade följande olika policyer:
      \begin{description}
        \item[basic8] Minst åtta tecken.

        \item[dictionary8] Minst åtta tecken, får inte finns med i ordlistan.

        \item[comprehensive8] Minst åtta tecken, måste innehålla små och stora 
          bokstäver, samt siffror och specialtecken.

        \item[basic16] Minst 16 tecken.
      \end{description}

      \pause{}

    \item Säkerheten var bäst hos basic16 (högst entropi), comprehensive8 var 
      näst bäst.

    \item Användbarhetsmässigt var basic16 bäst: användarna hade färre problem 
      att skriva in lösenordet och att komma ihåg det.
  \end{itemize}
\end{frame}

\begin{frame}
  \begin{example}[Real passwords]
    \begin{columns}
      \begin{column}{0.4\textwidth}
        \par
        Från~\cite{Oberheide2010bao}:
        \begin{itemize}
          \item 123456
          \item password
          \item 12345678
          \item qwerty
          \item abc123
        \end{itemize}
      \end{column}
      \begin{column}{0.4\textwidth}
        \par
        Från~\cite{Cluley2012twp}:
        \begin{itemize}
          \item 123456
          \item password
          \item welcome
          \item ninja
          \item abc123
        \end{itemize}
      \end{column}
    \end{columns}
  \end{example}
\end{frame}

% XXX move password cracking numbers to infotheory lecture
%\begin{frame}{Attackera ett konto eller alla}
%  \begin{block}{Ett givet konto}
%    \begin{itemize}
%      \item Svårt med lösenordsknäckning, måste testa \(|P|/2\) där \(P\) är 
%        mängden av alla lösenord.
%      \item Med phishing kallas detta \emph{spear phishing}.
%    \end{itemize}
%  \end{block}
%  \begin{block}{Alla konton på ett system}
%    \begin{itemize}
%      \item Avsevärt mycket enkare, närmare \(|P|/|U|\) där \(U\) är mängden av 
%        användare.
%      \item Kan använda phishing, räcker med att en användare faller för det.
%    \end{itemize}
%  \end{block}
%  \begin{block}{Alla konton på alla system}
%    \begin{itemize}
%      \item Knäck lösenord för ett enkelt system.
%      \item Phishing för ett system med dåliga policyer.
%      \item Sannolikt återanvänds lösenord i andra system.
%    \end{itemize}
%  \end{block}
%\end{frame}

\begin{frame}
  \begin{block}{Trusted path}
    Vi måste veta om vi kan lita på kommunikationskanalen.
  \end{block}

  \pause{}

  \begin{example}
    \begin{itemize}
      \item Är det ett riktigt tangentbord, eller är det utbytt mot ett som 
        sparar alla tangenttryckningar?
      \item Finns risken att det är en keylogger installerad?
    \end{itemize}
  \end{example}
\end{frame}

\subsection{Bättre lösningar?}

\begin{frame}
  \begin{itemize}
    \item Lösenord är har i sig dålig användbarhet.

    \item Finns olika metoder för att förbättra användbarheten.
      \begin{itemize}
        \item Single sign-on, exempelvis via Google eller Facebook.
        \item Spara alla lösenord krypterat och fyll automatiskt i dem på 
          webben.
          (Sabba inte detta alternativ med JavaScript.)
        \item Komplettera med koddosa.
      \end{itemize}

    \item Då har vi reducerat \(N\) lösenord till endast ett lösenord att komma 
      ihåg.

      \pause{}

    \item BankID verkar också vara en robust lösning.

  \end{itemize}
\end{frame}

\begin{frame}{BankID}
  \begin{itemize}
    \item Innebär att vi måste ha någonting: certifikatet.
    \item Vi måste veta någonting: lösenordet för certifikatet.
  \end{itemize}
\end{frame}

\begin{frame}{BankID hos Swedbank}
  % XXX get screenshots from BankID for Swedbank
  \begin{block}{Inloggning}
    I identify myself at:
    
    Swedbank och Sparbankerna
  \end{block}
  \begin{block}{Godkänna (signera) överföring}
    I sign at:
    
    Swedbank och Sparbankerna

    \vspace{1em}
    Text to be signed:

    Jag godkänner överföring med totalsumman 60,00 kr.
    Uppdraget lämnar jag till banken 2013-04-14 kl 21:25:43.
  \end{block}
\end{frame}

\begin{frame}{BankID hos Skatteverket}
  % XXX get screenshots from BankID for Skatteverket
  \begin{block}{Inloggning}
    I identify myself at:
    
    Skatteverket
  \end{block}
  \begin{block}{Signering av deklaration}
    I sign at:

    Skatteverket

    \vspace{1em}
    Text to be signed:

    Härmed undertecknar jag uppgifterna jag tidigare lämnat in.
  \end{block}
\end{frame}

\begin{frame}{BankID}
  \begin{itemize}
    \item Har separata mekanismer för identifiering och signering.
    \item Kan alltså inte lura användaren att signera en överföring vid 
      inloggning.
    \item Har ett användbart och pålitligt användargränssnitt.
%    \item Jämför med användargränssnittet till chippet på betalkortet (precis, 
%      det finns inget).
  \end{itemize}
\end{frame}

\begin{frame}
  \begin{itemize}
    \item Utforma autentisering för att användaren inte enkelt ska kunna bli 
      lurad!
    \item Om användaren blir lurad en gång ska inte det vara hela världen.
  \end{itemize}
\end{frame}

% XXX add slides on yubikey and lastpass

\begin{frame}
  \begin{itemize}
    \item Yubikey?
    \item LastPass?
  \end{itemize}
\end{frame}


