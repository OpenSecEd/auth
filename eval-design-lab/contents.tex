\title{%
  Lab: Evaluating and designing authentication
}
\author{Daniel Bosk}
\institute{%
  MIUN IST
  \and
  KTH EECS
}

\mode<article>{\maketitle}
\mode<presentation>{%
  \begin{frame}
    \maketitle
  \end{frame}
}

\mode*

\begin{abstract}
  Authentication has always been a central part of security.
An entity claims something, a property or an identity, authentication is about 
verifying or rejecting any such claim.
We will cover a few different ways to do authentication: the traditional 
something you know, something you have and something you are; but also look 
beyond.

Why we want to do this, and how we can accomplish this is treated in Chapter 
4 in~\cite{Gollmann2011cs}.
Anderson also treats this topic (Chapter 2 in~\cite{Anderson2008sea}), although 
in a wider perspective with less technical details.
When you have studied this material you should do exercises 4.2, 4.3, 4.4 and 
4.6 in~\cite{Gollmann2011cs}.

\end{abstract}


\section{Introduction}

Authentication is a vital part of most services.
The still most dominant authentication method is password based.
Passwords have their place in security, but they are over-used.
A person should need one or two passwords, not one or two \emph{hundred} 
passwords.
Password reuse is very bad for security, so there must be one strong, unique 
password for every service the user is interested in.
However, the need to keep track of hundreds of strong, unique passwords has 
developed the need for password managers.
Since no user remember any of these two hundred passwords, they could just as 
well be cryptographic keys, which would allow for more secure authentication 
methods.

When users must remember their credentials they tent to converge to using the 
same credentials for all services.
As pointed out above, such password reuse is bad.
But there is also a privacy issue.
Such systems use identity-based authentication.
Usually the email address is used as the identifier, and all of a user's 
actions can be attributed to this identity.
One issue with this practice of authentication is that it in many cases 
violates the principle of data minimization\footnote{%
  Data minimization requires that one uses only the bare minimum data that is 
  absolutely needed.
  This is one of the guiding principles of the \ac{GDPR}.
}.
For example, most services use identity-based authentication although it would 
do fine with another (less informative) attribute.

In this lab we are interested in exploring this topic from a security, privacy 
and usability perspective.

\subsection{Organization}

This lab is divided into two parts: evaluating authentication and designing 
authentication.
Each part is then divided over several sessions.
Work is done in groups.
During the session the groups work on a problem, present and discuss the 
results so that every group can learn the from the others.


\section{Evaluating authentication}

\paragraph{Design evaluation criteria}

During the first session, design evaluation criteria in in groups.
Remember to capture all perspectives: security, privacy and usability.
Present each group's criteria to the class.

\paragraph{Evaluate services}

For the second session, evaluate a few services (per group).
Pick one good example and one bad example to present to the class.
During the session, after the presentations, we will synthesize the results 
(lessons learned).


\section{Designing authentication}

During the third (and last) session, we will design proper authentication for 
some services.
Each group chooses a target service.
Then the groups work on designing the required authentication.
Finally, the results are presented for the whole class.


\section{Examination}

You must participate actively in the group work.


%%% REFERENCES %%%

\begin{frame}[allowframebreaks]
  \printbibliography
\end{frame}
