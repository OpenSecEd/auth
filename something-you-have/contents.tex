\mode*

% Since this a solution template for a generic talk, very little can
% be said about how it should be structured. However, the talk length
% of between 15min and 45min and the theme suggest that you stick to
% the following rules:  

% - Exactly two or three sections (other than the summary).
% - At *most* three subsections per section.
% - Talk about 30s to 2min per frame. So there should be between about
%   15 and 30 frames, all told.


\section{Something you have/know}

\subsection{Essence of authentication}

\begin{frame}
  \begin{remark}[Essence of authentication]
    \begin{itemize}
      \item Authentication is a challenge--response protocol.
      \item The verifier gives the prover a challenge.
      \item The prover responds to the verifier's challenge.
    \end{itemize}
  \end{remark}

  \pause

  \begin{example}[Passwords]
    \begin{itemize}
      \item Verifier's challenge: \enquote{what's the password?}
      \item Prover's response: the password.
    \end{itemize}
  \end{example}

  \pause

  \begin{remark}[Predictability]
    \begin{itemize}
      \item The challenge is predictable.
    \end{itemize}
  \end{remark}
\end{frame}

\begin{frame}
  \begin{question}
    \begin{itemize}
      \item Can we make the secrets more hard to guess?
      \item Can we have different challenges with different responses?
    \end{itemize}
  \end{question}
\end{frame}

\subsection{Cryptographic approaches}

\begin{frame}
  \begin{solution}
    \begin{itemize}
      \item Freshness is about challenge and response.
      \item Password-based authentication: the same challenge all the time.
      \item Improvement: random challenge, hard-to-guess response.
      \item We can do this with crypto.
    \end{itemize}
  \end{solution}
\end{frame}

\begin{frame}
  \begin{example}[Schnorr's protocol\footfullcite{Schnorr}]
    \begin{itemize}
      \item Prover's private key~\(x\), public key~\(g^x\).
      \item Prover wants to prove knowledge of \(x\) for \(g^x = y\).

        \pause

      \item Prover commits to randomness \(r\), by sending \(t = g^r\).

        \pause

      \item Verifier replies with randomly chosen challenge \(c\).

        \pause

      \item After receiving \(c\), prover replies with \(s = r + cx\).

        \pause{}

      \item Verifier accepts if \(g^s = t y^c\).
    \end{itemize}
  \end{example}
\end{frame}

\begin{frame}
  \begin{remark}
    \begin{itemize}
      \item We need password managers anyway, might just as well use \(x\).

        \pause

      \item It's more common to have helping devices (smartphones).

        \pause

      \item \(y = g^x\) is public, no more leaked secrets from server hacks.
    \end{itemize}
  \end{remark}
\end{frame}

\subsection{Anonymous Credentials}

\begin{frame}
  \begin{idea}
    \begin{itemize}
      \item Schnorr protocol is identity oriented.
      \item Generalize to other attributes.
    \end{itemize}
  \end{idea}
\end{frame}

\begin{frame}
  \begin{example}[Can do \dots]
    \begin{itemize}
      \item equalities
      \item inequalities
      \item conjunctions
      \item disjunctions
      \item knowledge of signatures
    \end{itemize}
  \end{example}
\end{frame}

\begin{frame}
  \begin{example}[Age limits]
    \begin{itemize}
      \item Bob wants to go see a film in cinema.
      \item Alice who works there wants to have proof of his age.
      
        \pause

      \item Bob has a certificate issued by someone Alice trusts.
      \item Bob doesn't want to show everything in the certificate.
      \item He proves that certificate says \(> 15\).
    \end{itemize}
  \end{example}
\end{frame}

\begin{frame}
  \begin{example}[Anonymous 
    Credentials\footfullcite{ElectronicIdentitiesNeedPrivateCredentials}]
    \begin{itemize}
      \item Makes heavy use of zero-knowledge proofs of knowledge.
      \item Can prove equalities, inequalities, knowledge, ownership, \dots
      \item Implementations and approaches:
        \begin{description}
          \item[Identity Mixer]
            \url{https://www.research.ibm.com/labs/zurich/idemix/}
          \item[U-Prove]
            \url{http://research.microsoft.com/en-us/projects/u-prove/}
          \item[AnonPass]
            \url{https://eprint.iacr.org/2013/317}
          \item[IRMA]
            \url{https://www.irmacard.org/irma/}
        \end{description}
    \end{itemize}
  \end{example}
\end{frame}


%%%%%%%%%%%%%%%%%%%%%%


\begin{frame}[allowframebreaks]
	\small
  \printbibliography{}
\end{frame}

