\mode*

% Since this a solution template for a generic talk, very little can
% be said about how it should be structured. However, the talk length
% of between 15min and 45min and the theme suggest that you stick to
% the following rules:  

% - Exactly two or three sections (other than the summary).
% - At *most* three subsections per section.
% - Talk about 30s to 2min per frame. So there should be between about
%   15 and 30 frames, all told.


\section{Something you have/know}

\subsection{Essence of authentication}

\begin{frame}
  \begin{remark}[Essence of authentication]
    \begin{itemize}
      \item Authentication is a challenge--response protocol.
      \item The verifier gives the prover a challenge.
      \item The prover responds to the verifier's challenge.
    \end{itemize}
  \end{remark}

  \pause

  \begin{example}[Passwords]
    \begin{itemize}
      \item Verifier's challenge: \enquote{what's the password?}
      \item Prover's response: the password.
    \end{itemize}
  \end{example}

  \pause

  \begin{remark}[Predictability]
    \begin{itemize}
      \item The challenge is predictable.
    \end{itemize}
  \end{remark}
\end{frame}

\begin{frame}
  \begin{question}
    \begin{itemize}
      \item Can we make the secrets more hard to guess?
      \item Can we have different challenges with different responses?
    \end{itemize}
  \end{question}
\end{frame}

\subsection{Cryptographic approaches}

\begin{frame}
  \begin{example}[Something you \dots]
    \begin{itemize}
      \item know (passwords)
      \item have (hardware tokens)
      \item are (passive biometrics)
      \item do (active biometrics)
    \end{itemize}
  \end{example}

  \pause{}

  \begin{remark}
    \begin{itemize}
      \item Do you \emph{know} a private key or do you \emph{have} one?
      \item A password you \emph{know}.
      \item A private key in a hardware token you \emph{have}.
      \item If the key is stored on your disk?
    \end{itemize}
  \end{remark}
\end{frame}

\begin{frame}
  \begin{solution}
    \begin{itemize}
      \item Freshness is about challenge and response.
      \item Password-based authentication: the same challenge all the time.
      \item Improvement: random challenge, hard-to-guess response.
    \end{itemize}
  \end{solution}
\end{frame}

\subsection{Anonymous Credentials}

\begin{frame}
  \begin{block}{The idea}
    \begin{itemize}
      \item What if Bob could convince Alice
        \begin{itemize}
          \item that he has a certificate saying he's older than 15,
          \item and is signed by someone Alice trusts.
        \end{itemize}
      \item Wouldn't that be awesome?
    \end{itemize}
  \end{block}
\end{frame}

\begin{frame}
  \begin{example}[Age limits]
    \begin{itemize}
      \item Bob wants to go see a film in cinema.
      \item Bob looks very young so Alice who works there wants to have proof 
        of his age.

        \pause{}

      \item Show physical ID, reveals name, exact date of birth, \dots
    \end{itemize}
  \end{example}

  \pause{}

  \begin{exercise}
    \begin{itemize}
      \item That's a bit overkill, right?
      \item What does Alice actually need to know?
      \item In what direction must we move to achieve that?
    \end{itemize}
  \end{exercise}
\end{frame}

\begin{frame}
  \begin{block}{What Alice needs?}
    She must be convinced that Bob is older than 15.
  \end{block}

  \pause{}

  \begin{alertblock}{How can she learn that?}
    \begin{enumerate}
      \item She has known Bob since he was born, so she knows.

        \pause{}

      \item She can ask someone \emph{she trusts} who knows Bob is older than 
        15.
    \end{enumerate}
  \end{alertblock}
\end{frame}

\begin{frame}
  \begin{alertblock}{But how can she do that?}
    \begin{enumerate}
      \item The trusted person who knows Bob is with Alice.

        \pause{}

      \item Alice can send a picture to the other person who verifies.
        \begin{itemize}
          \item This requires an \emph{authenticated} channel.
        \end{itemize}

        \pause{}

      \item The trusted person made a certificate for Bob showing that he's 
        older than 15.
        \begin{itemize}
          \item Alice must be able to \emph{verify} the certificate.
          \item Bob must bring this certificate with himself everywhere.
        \end{itemize}

    \end{enumerate}
  \end{alertblock}
\end{frame}

\begin{frame}
  \begin{alertblock}{Alice interacts with the trusted person}
    \begin{itemize}
      \item Gaah, but Bob doesn't want the trusted person (his parents) to know 
        he's at the cinema right now!
      \item It's a small cinema so they'll know which film he sees if they 
        learn when he's there.
    \end{itemize}
  \end{alertblock}

  \pause{}

  \begin{alertblock}{Alice reads and verifies the certificate}
    \begin{itemize}
      \item Phew, she accepted the note from his parents.
      \item But now Alice learned all those embarrassing things in 
        there.
        \begin{itemize}
          \item And Bob who has a crush on Alice \dots
        \end{itemize}
    \end{itemize}
  \end{alertblock}
\end{frame}

\begin{frame}
  \begin{block}{The idea}
    \begin{itemize}
      \item What if Bob could convince Alice
        \begin{itemize}
          \item that he has a certificate saying he's older than 15,
          \item and is signed by someone Alice trusts.
        \end{itemize}
      \item Wouldn't that be awesome?
    \end{itemize}
  \end{block}
\end{frame}

\begin{frame}
  \begin{example}[Anonymous 
    Credentials\footfullcite{ElectronicIdentitiesNeedPrivateCredentials}]
    \begin{itemize}
      \item Makes heavy use of zero-knowledge proofs of knowledge.
      \item Can prove equalities, inequalities, knowledge, ownership, \dots
      \item Implementations and approaches:
        \begin{description}
          \item[Identity Mixer]
            \url{https://www.research.ibm.com/labs/zurich/idemix/}
          \item[U-Prove]
            \url{http://research.microsoft.com/en-us/projects/u-prove/}
          \item[AnonPass]
            \url{https://eprint.iacr.org/2013/317}
          \item[IRMA]
            \url{https://www.irmacard.org/irma/}
        \end{description}
    \end{itemize}
  \end{example}
\end{frame}


%%%%%%%%%%%%%%%%%%%%%%


\begin{frame}[allowframebreaks]
	\small
  \printbibliography{}
\end{frame}

