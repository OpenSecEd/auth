\mode*

% Since this a solution template for a generic talk, very little can
% be said about how it should be structured. However, the talk length
% of between 15min and 45min and the theme suggest that you stick to
% the following rules:  

% - Exactly two or three sections (other than the summary).
% - At *most* three subsections per section.
% - Talk about 30s to 2min per frame. So there should be between about
%   15 and 30 frames, all told.


\section{Something you know}

\subsection{\enquote{Proof of knowledge}}

\begin{frame}
  \begin{idea}[Something you know]
    \begin{itemize}
      \item We have a prover and a verifier.
      \item Prover must convince verifier he knows some secret.
    \end{itemize}
  \end{idea}

  \begin{idea}[Password]
    \begin{itemize}
      \item Prover and verifier shares a secret value.
      \item Prover tells verifier the value to convince the verifier.
    \end{itemize}
  \end{idea}
\end{frame}

\begin{frame}
  \begin{remark}
    \begin{itemize}
      \item If the adversary learns the secret, he can convince the verifier he 
        is the prover.
    \end{itemize}
  \end{remark}

  \begin{example}
    \begin{itemize}
      \item Adversary might \enquote{overhear the conversation}.
      \item Adversary might \enquote{trick} the prover to reveal the secret.
      \item Adversary might guess the secret.
    \end{itemize}
  \end{example}
\end{frame}

\subsection{Guessing secrets}

\begin{frame}
  \begin{idea}
    \begin{itemize}
      \item The secret~\(x\) is chosen from a probability distribution.
      \item The probability of guessing correctly is \(\Pr[X = x]\).
    \end{itemize}
  \end{idea}

  \begin{remark}
    \begin{itemize}
      \item The question is what the probability distribution is.
    \end{itemize}
  \end{remark}
\end{frame}

\begin{frame}
  \begin{example}[Cryptographic keys]
    \begin{itemize}
      \item The distribution is \emph{very} close to the uniform distribution.
      \item \Ie \(\Pr[X = x] = 1/n\), where \(X\) can take \(n\) possible 
        values.
      \item In crypto, normally \(n = 2^{128}\).
    \end{itemize}
  \end{example}

  \pause

  \begin{example}[Password]
    \begin{itemize}
      \item Distribution of passwords is affected by so many factors.
      \item The individual, situation, password policies, \etc.
    \end{itemize}
  \end{example}
\end{frame}

\begin{frame}
  \begin{figure}
    \includegraphics[height=0.9\textheight]{distributions.png}
  \end{figure}
\end{frame}

\begin{frame}
  \begin{idea}[Guessing passwords]
    \begin{itemize}
      \item Find a way to approximate the distribution.
    \end{itemize}
  \end{idea}

  \begin{example}[Basic guessing]
    \begin{itemize}
      \item Using dictionaries of words.
      \item Adapt guesses to password policy, if known.
      \item \dots
    \end{itemize}
  \end{example}
\end{frame}

\begin{frame}
  \begin{example}[Improved guessing]
    \begin{itemize}
      \item Use leaked passwords as guesses.
      \item Take grammar into account, depending on the password 
        type~\cite{Bonneau2012ghs,Bonneau2012lpo}.
    \end{itemize}
  \end{example}

  \pause

  \begin{example}[Machine learning]
    \begin{itemize}
      \item Use machine learning~\cite{JohnTheRipper,OMEN,WeirPCFG}.
      \item Train algorithm on leaked password databases.
      \item Generate list of password-looking guesses.
    \end{itemize}
  \end{example}
\end{frame}

\begin{frame}
  \begin{remark}
    \begin{itemize}
      \item This is relevant when the user has chosen a password.
      \item In many situations it's not.
    \end{itemize}
  \end{remark}

  \pause{}

  \begin{example}
    \begin{itemize}
      \item There are many devices with default passwords.
      \item \Eg home routers, \dots
    \end{itemize}
  \end{example}
\end{frame}

\begin{frame}
  \begin{example}[Mirai botnet~\cite{MiraiAnalysis}]
    \begin{itemize}
      \item Botnet infecting primarily surveillance cameras and home routers.
      \item Attempts default passwords and other vulnerabilities.
      \item Managed the largest \ac{DDoS} attack hitherto.
    \end{itemize}
  \end{example}

  \begin{remark}
    \begin{itemize}
      \item These default passwords have very high probability.
    \end{itemize}
  \end{remark}
\end{frame}

\begin{frame}
  \begin{idea}[Autogenerate passwords]
    \begin{itemize}
      \item Generate passwords for users.
      \item This yields a uniform distribution.
    \end{itemize}
  \end{idea}

  \begin{remark}[Usability]
    \begin{itemize}
      \item This will likely reduce security by use of post-it notes.
      \item Not a problem for a home router.
      \item Otherwise: will require password managers.
    \end{itemize}
  \end{remark}
\end{frame}

\begin{frame}
  \begin{idea}[Password policy]
    \begin{itemize}
      \item Introduce rules to affect how users choose passwords.
      \item We require upper, lower case, numbers, special characters.
      \item Then passwords will be more uniform-looking.
    \end{itemize}
  \end{idea}

  \begin{remark}[Usability]
    \begin{itemize}
      \item This has been proven a bad idea.
      \item Research has estimated the distribution under various 
        policies~\cite{Komanduri2011opa}.
      \item Better to only require length.
    \end{itemize}
  \end{remark}
\end{frame}

\begin{frame}
  \begin{idea}[Password ageing]
    \begin{itemize}
      \item Let passwords age and expire.
      \item Then users change them frequently.
      \item If it takes six months to guess and we change every three \dots
    \end{itemize}
  \end{idea}

  \begin{remark}[Usability]
    \begin{itemize}
      \item This has been proven a bad idea.
      \item Annoying with too short intervals.
      \item Will reduce security once users introduce systems to remember their 
        last changed password.
    \end{itemize}
  \end{remark}
\end{frame}

\begin{frame}
  \begin{itemize}
    \item \Textcite{NIST-passwd-guide} summarizes the current recommendations.
    \item At least 10 characters.
    \item Force renewal only after security breach.
  \end{itemize}
\end{frame}


\subsection{Online or offline?}

\begin{frame}
  \begin{definition}[Online]
    \begin{itemize}
      \item The adversary must interact with the system for each guess.
    \end{itemize}
  \end{definition}

  \pause

  \begin{definition}[Offline]
    \begin{itemize}
      \item The adversary can verify the guess himself.
    \end{itemize}
  \end{definition}
\end{frame}

\begin{frame}
  \begin{example}[Online]
    \begin{itemize}
      \item Guessing the password of a Google account.
      \item Must submit each guess to Google.
    \end{itemize}
  \end{example}

  \pause

  \begin{example}[Offline]
    \begin{itemize}
      \item Guessing the password of an encrypted file.
      \item For each guess, try to decrypt.
    \end{itemize}
  \end{example}
\end{frame}

\begin{frame}
  \begin{solution}[Rate limiting]
    \begin{itemize}
      \item For online guessing, rate limit the attempts.
      \item This makes guessing too slow.
    \end{itemize}
  \end{solution}

  \pause

  \begin{remark}
    \begin{itemize}
      \item This works for targeted attacks.
      \item Introduces possibility for denial-of-service.
    \end{itemize}
  \end{remark}
\end{frame}

\begin{frame}[fragile]
  \begin{exercise}
    \begin{itemize}
      \item How will rate limiting affect these?
    \end{itemize}

    \begin{minipage}{0.45\textwidth}
      \begin{lstlisting}
for u in users:
  for p in passwds:
    try_login(u, p)
      \end{lstlisting}
    \end{minipage}
    \hfill
    \begin{minipage}{0.45\textwidth}
      \begin{lstlisting}
for p in passwds:
  for u in users:
    try_login(u, p)
      \end{lstlisting}
    \end{minipage}
  \end{exercise}
\end{frame}

\begin{frame}
  \begin{remark}
    \begin{itemize}
      \item Maybe we don't care about which user.
      \item If a password is common, then it's likely that \emph{some} user 
        chose it.
      \item And if we try one password for each user we might not trigger the 
        rate limiting.
    \end{itemize}
  \end{remark}
\end{frame}

\begin{frame}
  \begin{remark}[Offline]
    \begin{itemize}
      \item Consider data which is encrypted with a password.
      \item You cannot change a password for data that is already stolen.
      \item You cannot limit the number of attempts either.
      \item You can just control the guessability of the password.
    \end{itemize}
  \end{remark}
\end{frame}

\subsection{Storing secrets}

\begin{frame}
  \begin{exercise}
    \begin{itemize}
      \item We now have data to authenticate users.
      \item How do we store this data?
      \item What problems do you see?
    \end{itemize}
  \end{exercise}
\end{frame}

\begin{frame}
  \begin{example}[Password-based authentication]
    \begin{itemize}
      \item Traditionally, there is a password file (or database).
      \item This contains all users' passwords.

        \pause{}

      \item If someone copies this data, he or she could impersonate any user 
        in the system.
    \end{itemize}
  \end{example}
\end{frame}

\begin{frame}
  \begin{solution}[Passwords]
    \begin{itemize}
      \item We want to compare user-entered and stored password.
      \item We do an irreversible one-way transformation on both.
      \item Then they are still comparable.
      \item The preimage cannot be gained from storage.
    \end{itemize}
  \end{solution}

  \pause{}

  \begin{example}
    \begin{itemize}
      \item Cryptographic hash function \(h\colon \bin^*\to \bin^n\).
      \item On registration, store \(h(p)\).
      \item User authenticates with \(p'\), check if \(h(p') \stackrel{?}{=}  
          h(p)\) equals what we stored.
    \end{itemize}
  \end{example}
\end{frame}

\begin{frame}
  \begin{remark}
    \begin{itemize}
      \item Consider guessing again.
      \item The used password space is small.
      \item We only need to evaluate a subset: \(h\colon 
          \bin^{\color{red}{m}}\to \bin^n\).
      \item With faster computers we can guess a lot.
    \end{itemize}
  \end{remark}

  \pause{}

  \begin{solution}
    \begin{itemize}
      \item Choose \(h\) to be slow to compute.
      \item E.g.\ iterate it over itself 10\,000 times.
      \item This will slow down guessing attacks.
    \end{itemize}
  \end{solution}
\end{frame}

\begin{frame}
  \begin{remark}
    \begin{itemize}
      \item The password file structure reveals if two users have the same 
        password.
      \item Can guess the password for all users at once:
        \begin{enumerate}
          \item Make a guess, compute the hash.
          \item Check if it matches \emph{any} user's password.
        \end{enumerate}
    \end{itemize}
  \end{remark}

  \pause{}

  \begin{solution}
    \begin{itemize}
      \item Add a \emph{salt}: a small random value (e.g.\ 128 bits) unique for 
        each user.
      \item Include this value in the computation of the password hash.
      \item Now all hashes will be unique.
    \end{itemize}
  \end{solution}
\end{frame}

\begin{frame}
  \begin{remark}
    \begin{itemize}
      \item The salt is not a secret, it's just unique.
      \item It can be stored in plain text along with the password hash.
    \end{itemize}
  \end{remark}
\end{frame}

\begin{frame}
  \begin{example}
    \begin{itemize}
      \item bcrypt~\cite{bcrypt} implements all this functionality.
      \item It should also be available in most languages and libraries.
    \end{itemize}
  \end{example}
\end{frame}

\begin{frame}
  \begin{exercise}
    \begin{itemize}
      \item This is a problem when the authentication mechanism faces the 
        Internet.

      \item \Eg home routers where the admin interface only faces the local 
        network should be fine.

      \item (The same if we have a white list of addresses allowed access.)

      \item What do you think?
    \end{itemize}
  \end{exercise}
\end{frame}


\section{Something you know/have}

\subsection{Cryptographic approaches}

\begin{frame}
  \begin{example}[Something you \dots]
    \begin{itemize}
      \item know (passwords)
      \item have (hardware tokens)
      \item are (passive biometrics)
      \item do (active biometrics)
    \end{itemize}
  \end{example}

  \pause{}

  \begin{remark}
    \begin{itemize}
      \item Do you \emph{know} a private key or do you \emph{have} one?
      \item A password you \emph{know}.
      \item A private key in a hardware token you \emph{have}.
      \item If the key is stored on your disk?
    \end{itemize}
  \end{remark}
\end{frame}

\begin{frame}
  \begin{example}[Passive biometrics]
    \begin{itemize}
      \item Fingerprint
      \item Irises
      \item DNA
    \end{itemize}
  \end{example}

  \pause{}

  \begin{example}[Active biometrics]
    \begin{itemize}
      \item Typing speed
      \item Cursor movement
      \item Web surfing behaviour
      \item Pressure points in signature
    \end{itemize}
  \end{example}
\end{frame}

\begin{frame}
  \begin{exercise}
    \begin{itemize}
      \item What about something you are (passive biometrics)?
      \item When is that more than merely a password?
    \end{itemize}
  \end{exercise}
\end{frame}

\begin{frame}
  \begin{example}[Fingerprints for iPhones]
    \begin{itemize}
      \item The iPhone can trust its built-in fingerprint reader.
      \item We know that we read an actual finger.
      \item We know when we read it, so we have freshness.
    \end{itemize}
  \end{example}

  \pause{}

  \begin{example}[Fingerprints for web services]
    \begin{itemize}
      \item Fingerprint reader connected to laptop.
      \item Browser scans and sends fingerprint to server.
      \item Anyone could send that data, without the reader.
      \item We lack freshness guarantees.
    \end{itemize}
  \end{example}
\end{frame}

\begin{frame}
  \begin{remark}
    \begin{itemize}
      \item We need freshness.
      \item We must store the fingerprint somewhere, to compare.
      \item Someone can copy that data.
      \item Without freshness they can use it --- as a password.
    \end{itemize}
  \end{remark}

  \pause{}

  \begin{remark}
    \begin{itemize}
      \item Similar for interactive biometrics.
      \item Record and replay.
    \end{itemize}
  \end{remark}
\end{frame}

\begin{frame}
  \begin{exercise}
    \begin{itemize}
      \item How can we ensure freshness?
      \item What data do we need to store for this?
      \item How can we secure that data?
    \end{itemize}
  \end{exercise}
\end{frame}

\begin{frame}
  \begin{solution}
    \begin{itemize}
      \item Freshness is about challenge and response.
      \item Password-based authentication: the same challenge all the time.
      \item Improvement: random challenge, hard-to-guess response.
    \end{itemize}
  \end{solution}
\end{frame}

\subsection{Anonymous Credentials}

\begin{frame}
  \begin{example}[Age limits]
    \begin{itemize}
      \item Bob wants to go see a film in cinema.
      \item Bob looks very young so Alice who works there wants to have proof 
        of his age.

        \pause{}

      \item Show physical ID, reveals name, exact date of birth, \dots
    \end{itemize}
  \end{example}

  \pause{}

  \begin{exercise}
    \begin{itemize}
      \item That's a bit overkill, right?
      \item What does Alice actually need to know?
      \item In what direction must we move to achieve that?
    \end{itemize}
  \end{exercise}
\end{frame}

\begin{frame}
  \begin{block}{What Alice needs?}
    She must be convinced that Bob is older than 15.
  \end{block}

  \pause{}

  \begin{alertblock}{How can she learn that?}
    \begin{enumerate}
      \item She has known Bob since he was born, so she knows.

        \pause{}

      \item She can ask someone \emph{she trusts} who knows Bob is older than 
        15.
    \end{enumerate}
  \end{alertblock}
\end{frame}

\begin{frame}
  \begin{alertblock}{But how can she do that?}
    \begin{enumerate}
      \item The trusted person who knows Bob is with Alice.

        \pause{}

      \item Alice can send a picture to the other person who verifies.
        \begin{itemize}
          \item This requires an \emph{authenticated} channel.
        \end{itemize}

        \pause{}

      \item The trusted person made a certificate for Bob showing that he's 
        older than 15.
        \begin{itemize}
          \item Alice must be able to \emph{verify} the certificate.
          \item Bob must bring this certificate with himself everywhere.
        \end{itemize}

    \end{enumerate}
  \end{alertblock}
\end{frame}

\begin{frame}
  \begin{alertblock}{Alice interacts with the trusted person}
    \begin{itemize}
      \item Gaah, but Bob doesn't want the trusted person (his parents) to know 
        he's at the cinema right now!
      \item It's a small cinema so they'll know which film he sees if they 
        learn when he's there.
    \end{itemize}
  \end{alertblock}

  \pause{}

  \begin{alertblock}{Alice reads and verifies the certificate}
    \begin{itemize}
      \item Phew, she accepted the note from his parents.
      \item But now Alice learned all those embarrassing things in 
        there.
        \begin{itemize}
          \item And Bob who has a crush on Alice \dots
        \end{itemize}
    \end{itemize}
  \end{alertblock}
\end{frame}

\begin{frame}
  \begin{block}{The idea}
    \begin{itemize}
      \item What if Bob could convince Alice
        \begin{itemize}
          \item that he has a certificate saying he's older than 15,
          \item and is signed by someone Alice trusts.
        \end{itemize}
      \item Wouldn't that be awesome?
    \end{itemize}
  \end{block}
\end{frame}

\begin{frame}
  \begin{example}[Anonymous 
    Credentials\footfullcite{ElectronicIdentitiesNeedPrivateCredentials}]
    \begin{itemize}
      \item Makes heavy use of zero-knowledge proofs of knowledge.
      \item Can prove equalities, inequalities, knowledge, ownership, \dots
      \item Implementations and approaches:
        \begin{description}
          \item[Identity Mixer]
            \url{https://www.research.ibm.com/labs/zurich/idemix/}
          \item[U-Prove]
            \url{http://research.microsoft.com/en-us/projects/u-prove/}
          \item[AnonPass]
            \url{https://eprint.iacr.org/2013/317}
          \item[IRMA]
            \url{https://www.irmacard.org/irma/}
        \end{description}
    \end{itemize}
  \end{example}
\end{frame}


%%%%%%%%%%%%%%%%%%%%%%


\begin{frame}[allowframebreaks]
	\small
  \printbibliography{}
\end{frame}

