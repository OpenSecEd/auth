\mode*

% Since this a solution template for a generic talk, very little can
% be said about how it should be structured. However, the talk length
% of between 15min and 45min and the theme suggest that you stick to
% the following rules:  

% - Exactly two or three sections (other than the summary).
% - At *most* three subsections per section.
% - Talk about 30s to 2min per frame. So there should be between about
%   15 and 30 frames, all told.


\section{User--machine authentication}

\subsection{What's the problem?}

\begin{frame}
  \begin{example}[Identifiers]
    \begin{itemize}
      \item Username or User ID
      \item The person who opened the account
      \item Personal Identification Number (Swe.\ personnummer, Eng.\ Social 
        Security Number)

        \pause{}

      \item Fingerprint
      \item Iris scan
      \item DNA sequence \dots
    \end{itemize}
  \end{example}

  \pause{}

  \begin{exercise}
    Any other methods of user identification that you have encountered?
  \end{exercise}
\end{frame}

\subsection{Some approaches}

\begin{frame}
  \begin{example}[Methods for user authentication]
    \begin{itemize}
      \item Something you \emph{know}

        \pause{}

      \item Something you \emph{have}

        \pause{}

      \item \emph{Where} you are

        \pause{}

      \item \emph{Who} you are
      \item What you \emph{do}
    \end{itemize}
  \end{example}
\end{frame}

\subsection{Multi-factor user-authentication}

\begin{frame}
  \begin{definition}[Multi-factor authentication]
    \begin{itemize}
      \item Combine two or more methods of authentication.
    \end{itemize}
  \end{definition}

  \begin{example}[Methods for user authentication]
    \begin{itemize}
      \item Something you \emph{know}
      \item Something you \emph{have}
      \item \emph{Where} you are
      \item \emph{Who} you are
      \item What you \emph{do}
    \end{itemize}
  \end{example}
\end{frame}

\begin{frame}
  \begin{example}[Single-factor authentication]
    \begin{description}
      \item[Identification] Username or similar
      \item[Authentication] Something you know, i.e.\ a password
    \end{description}
  \end{example}

  \pause{}

  \begin{example}[Multi-factor authentication]
    \begin{description}
      \item[Identification] Username or similar
      \item[Authentication] Something you know together with something you 
        have, e.g.\ password and mobile phone
    \end{description}
  \end{example}
\end{frame}

\subsection{Time of check, time of use}

\begin{frame}
  \begin{exercise}
    \begin{itemize}
      \item Whenever we authenticate a user, we do this for a purpose.

      \item When does this authentication take place in relation to when we make 
        use of it?
    \end{itemize}
  \end{exercise}
\end{frame}

\begin{frame}
  \begin{example}
    \begin{itemize}
      \item Usually we authenticate a user in the beginning of a session, e.g.\ 
        at login.

      \item Equally often we assume the user is authenticated during the entire 
        session, even when fetching coffee, going by the printer -- or even when 
        out to lunch.

      \item Who knows what happens when the user is away from the computer, one 
        thing is for sure: the computer will not know the difference!

    \end{itemize}
  \end{example}
\end{frame}

\begin{frame}
  \begin{solution}
    \begin{itemize}
      \item This problem can be solved with \emph{repeated authentication}.

      \item We could lock our system, either manually or by time-out.

      \item We could also authenticate anew when we need to do something 
        requiring more privileges, and if it has been a while since last time -- 
        compare with sudo(8).
    \end{itemize}
  \end{solution}

  \pause{}

  \begin{remark}
    \begin{itemize}
      \item What we actually need is \emph{continuous} authentication.
    \end{itemize}
  \end{remark}
\end{frame}

\section[Securing]{Securing Authentication}

\subsection{Guessing Passwords}

\begin{frame}
  \begin{itemize}
    \item Guessing passwords is like searching for a needle in a haystack.

      \pause{}

    \item (Un)fortunately, the needle is placed by a human --- not uniformly 
      randomly!
    \item This makes guessing easier.
    \item Human-chosen passwords will only occupy parts of the password space.
  \end{itemize}
\end{frame}

\begin{frame}
  \begin{itemize}
    \item The effort is a spectrum.

    \item It ranges from brute-force exhaustive search \dots

      \pause{}

    \item \dots via \enquote{educated guessing} \dots

      \pause{}

    \item \dots to getting the password from the user directly.
  \end{itemize}
\end{frame}

\begin{frame}
  \begin{example}[Basic guessing]
    \begin{itemize}
      \item Using dictionaries of words.
      \item Adapt to guesses to password policy, if known.
      \item \dots
    \end{itemize}
  \end{example}

  \pause{}

  \begin{example}[Improved guessing]
    Take grammar into account, depending on the password 
    type~\cite{Bonneau2012ghs,Bonneau2012lpo}.
  \end{example}
\end{frame}

\begin{frame}
  \begin{example}[Learn from humans]
    \begin{itemize}
      \item Use machine learning~\cite{JohnTheRipper,OMEN,WeirPCFG}.
      \item Train algorithm on leaked password databases.
      \item Generate list of password-looking guesses.
    \end{itemize}
  \end{example}
\end{frame}

\begin{frame}
  \begin{remark}
    \begin{itemize}
      \item This is relevant when the user has chosen a password.
      \item In the majority of situations it's not.
    \end{itemize}
  \end{remark}

  \pause{}

  \begin{example}
    \begin{itemize}
      \item There are many devices with default passwords.
      \item E.g.\ home routers, \dots
    \end{itemize}
  \end{example}
\end{frame}

\begin{frame}
  \begin{remark}
    \begin{itemize}
      \item The problem of default passwords has increased recently.
      \item Home routers, web cameras are open to attack.
    \end{itemize}
  \end{remark}

  \pause{}

  \begin{example}[Mirai botnet~\cite{MiraiAnalysis}]
    \begin{itemize}
      \item Botnet infecting primarily surveillance cameras.
      \item Attempts default passwords and other vulnerabilities.
      \item Managed the largest \ac{DDoS} attack hitherto.
    \end{itemize}
  \end{example}
\end{frame}

\begin{frame}
  \begin{exercise}
    \begin{itemize}
      \item This is a problem when the authentication mechanism faces the 
        Internet.

      \item E.g.\ home routers where the admin interface only faces the local 
        network should be fine.

      \item (The same if we have a white list of addresses allowed access.)

      \item What do you think?
    \end{itemize}
  \end{exercise}
\end{frame}

\begin{frame}
  \begin{example}[Autogenerate passwords]
    \begin{itemize}
      \item Generate passwords for users.
      \item This will likely reduce security by use of post-it notes.
      \item Not a problem for a home router.
    \end{itemize}
  \end{example}
\end{frame}

\begin{frame}
  \begin{example}[Password ageing]
    \begin{itemize}
      \item Let passwords age and expire.
      \item Annoying with too short intervals.
      \item Will reduce security once users introduce systems to remember their 
        last changed password.
      \item Just an expiration date for the generated one, infinite selected by
        user.
    \end{itemize}
  \end{example}
\end{frame}

\begin{frame}
  \begin{example}[Rate limiting]
    \begin{itemize}
      \item Remove online guessing by limited login attempts.
      \item Introduces the possibility of denial of service.
    \end{itemize}
  \end{example}
\end{frame}

\begin{frame}
  \begin{remark}[Offline data]
    \begin{itemize}
      \item Consider data which is encrypted with a password.
      \item You cannot change a password for data that is already stolen.
      \item You cannot limit the number of attempts either.
      \item You can just control the guessability of the password.
    \end{itemize}
  \end{remark}
\end{frame}

\subsection{The Password File}

\begin{frame}
  \begin{exercise}
    \begin{itemize}
      \item We now have data to authenticate users.
      \item How do we store this data?
      \item What problems do you see?
    \end{itemize}
  \end{exercise}
\end{frame}

\begin{frame}
  \begin{example}[Password-based authentication]
    \begin{itemize}
      \item Traditionally, there is a password file (or database).
      \item This contains all users' passwords.

        \pause{}

      \item If someone copies this data, he or she could impersonate any user 
        in the system.
    \end{itemize}
  \end{example}
\end{frame}

\begin{frame}
  \begin{solution}[Passwords]
    \begin{itemize}
      \item We want to compare user-entered and stored password.
      \item We do an irreversible one-way transformation on both.
      \item Then they are still comparable.
      \item The preimage cannot be gained from storage.
    \end{itemize}
  \end{solution}

  \pause{}

  \begin{example}
    \begin{itemize}
      \item Cryptographic hash function \(h\colon \bin^*\to \bin^n\).
      \item On registration, store \(h(p)\).
      \item User authenticates with \(p'\), check if \(h(p') \stackrel{?}{=}  
          h(p)\) equals what we stored.
    \end{itemize}
  \end{example}
\end{frame}

\begin{frame}
  \begin{remark}
    \begin{itemize}
      \item Consider guessing again.
      \item The used password space is small.
      \item We only need to evaluate a subset: \(h\colon 
          \bin^{\color{red}{m}}\to \bin^n\).
      \item With faster computers we can guess a lot.
    \end{itemize}
  \end{remark}

  \pause{}

  \begin{solution}
    \begin{itemize}
      \item Choose \(h\) to be slow to compute.
      \item E.g.\ iterate it over itself 10\,000 times.
      \item This will slow down guessing attacks.
    \end{itemize}
  \end{solution}
\end{frame}

\begin{frame}
  \begin{remark}
    \begin{itemize}
      \item The password file structure reveals if two users have the same 
        password.
      \item Can guess the password for all users at once:
        \begin{enumerate}
          \item Make a guess, compute the hash.
          \item Check if it matches \emph{any} user's password.
        \end{enumerate}
    \end{itemize}
  \end{remark}

  \pause{}

  \begin{solution}
    \begin{itemize}
      \item Add a \emph{salt}: a small random value (e.g.\ 128 bits) unique for 
        each user.
      \item Include this value in the computation of the password hash.
      \item Now all hashes will be unique.
    \end{itemize}
  \end{solution}
\end{frame}

\begin{frame}
  \begin{remark}
    \begin{itemize}
      \item The salt is not a secret, it's just unique.
      \item It can be stored in plain text along with the password hash.
    \end{itemize}
  \end{remark}
\end{frame}

\begin{frame}
  \begin{example}
    \begin{itemize}
      \item bcrypt~\cite{bcrypt} implements all this functionality.
      \item It should also be available in most languages and libraries.
    \end{itemize}
  \end{example}
\end{frame}

\subsection{Alternative Approaches}

\begin{frame}
  \begin{example}[Something you \dots]
    \begin{itemize}
      \item know (passwords)
      \item have (hardware tokens)
      \item are (passive biometrics)
      \item do (active biometrics)
    \end{itemize}
  \end{example}

  \pause{}

  \begin{remark}
    \begin{itemize}
      \item Do you \emph{know} a private key or do you \emph{have} one?
      \item A password you \emph{know}.
      \item A private key in a hardware token you \emph{have}.
      \item If the key is stored on your disk?
    \end{itemize}
  \end{remark}
\end{frame}

\begin{frame}
  \begin{example}[Passive biometrics]
    \begin{itemize}
      \item Fingerprint
      \item Irises
      \item DNA
    \end{itemize}
  \end{example}

  \pause{}

  \begin{example}[Active biometrics]
    \begin{itemize}
      \item Typing speed
      \item Cursor movement
      \item Web surfing behaviour
      \item Pressure points in signature
    \end{itemize}
  \end{example}
\end{frame}

\begin{frame}
  \begin{exercise}
    \begin{itemize}
      \item What about something you are (passive biometrics)?
      \item When is that more than merely a password?
    \end{itemize}
  \end{exercise}
\end{frame}

\begin{frame}
  \begin{example}[Fingerprints for iPhones]
    \begin{itemize}
      \item The iPhone can trust its built-in fingerprint reader.
      \item We know that we read an actual finger.
      \item We know when we read it, so we have freshness.
    \end{itemize}
  \end{example}

  \pause{}

  \begin{example}[Fingerprints for web services]
    \begin{itemize}
      \item Fingerprint reader connected to laptop.
      \item Browser scans and sends fingerprint to server.
      \item Anyone could send that data, without the reader.
      \item We lack freshness guarantees.
    \end{itemize}
  \end{example}
\end{frame}

\begin{frame}
  \begin{remark}
    \begin{itemize}
      \item We need freshness.
      \item We must store the fingerprint somewhere, to compare.
      \item Someone can copy that data.
      \item Without freshness they can use it --- as a password.
    \end{itemize}
  \end{remark}

  \pause{}

  \begin{remark}
    \begin{itemize}
      \item Similar for interactive biometrics.
      \item Record and replay.
    \end{itemize}
  \end{remark}
\end{frame}

\begin{frame}
  \begin{exercise}
    \begin{itemize}
      \item How can we ensure freshness?
      \item What data do we need to store for this?
      \item How can we secure that data?
    \end{itemize}
  \end{exercise}
\end{frame}

\begin{frame}
  \begin{solution}
    \begin{itemize}
      \item Freshness is about challenge and response.
      \item Password-based authentication: the same challenge all the time.
      \item Improvement: random challenge, hard-to-guess response.
    \end{itemize}
  \end{solution}
\end{frame}

\subsection{Anonymous Credentials}

\begin{frame}
  \begin{example}[Age limits]
    \begin{itemize}
      \item Bob wants to go see a film in cinema.
      \item Bob looks very young so Alice who works there wants to have proof 
        of his age.

        \pause{}

      \item Show physical ID, reveals name, exact date of birth, \dots
    \end{itemize}
  \end{example}

  \pause{}

  \begin{exercise}
    \begin{itemize}
      \item That's a bit overkill, right?
      \item What does Alice actually need to know?
      \item In what direction must we move to achieve that?
    \end{itemize}
  \end{exercise}
\end{frame}

\begin{frame}
  \begin{block}{What Alice needs?}
    She must be convinced that Bob is older than 15.
  \end{block}

  \pause{}

  \begin{alertblock}{How can she learn that?}
    \begin{enumerate}
      \item She has known Bob since he was born, so she knows.

        \pause{}

      \item She can ask someone \emph{she trusts} who knows Bob is older than 
        15.
    \end{enumerate}
  \end{alertblock}
\end{frame}

\begin{frame}
  \begin{alertblock}{But how can she do that?}
    \begin{enumerate}
      \item The trusted person who knows Bob is with Alice.

        \pause{}

      \item Alice can send a picture to the other person who verifies.
        \begin{itemize}
          \item This requires an \emph{authenticated} channel.
        \end{itemize}

        \pause{}

      \item The trusted person made a certificate for Bob showing that he's 
        older than 15.
        \begin{itemize}
          \item Alice must be able to \emph{verify} the certificate.
          \item Bob must bring this certificate with himself everywhere.
        \end{itemize}

    \end{enumerate}
  \end{alertblock}
\end{frame}

\begin{frame}
  \begin{alertblock}{Alice interacts with the trusted person}
    \begin{itemize}
      \item Gaah, but Bob doesn't want the trusted person (his parents) to know 
        he's at the cinema right now!
      \item It's a small cinema so they'll know which film he sees if they 
        learn when he's there.
    \end{itemize}
  \end{alertblock}

  \pause{}

  \begin{alertblock}{Alice reads and verifies the certificate}
    \begin{itemize}
      \item Phew, she accepted the note from his parents.
      \item But now Alice learned all those embarrassing things in 
        there.
        \begin{itemize}
          \item And Bob who has a crush on Alice \dots
        \end{itemize}
    \end{itemize}
  \end{alertblock}
\end{frame}

\begin{frame}
  \begin{block}{The idea}
    \begin{itemize}
      \item What if Bob could convince Alice
        \begin{itemize}
          \item that he has a certificate saying he's older than 15,
          \item and is signed by someone Alice trusts.
        \end{itemize}
      \item Wouldn't that be awesome?
    \end{itemize}
  \end{block}
\end{frame}

\begin{frame}
  \begin{example}[Anonymous 
    Credentials\footfullcite{ElectronicIdentitiesNeedPrivateCredentials}]
    \begin{itemize}
      \item Makes heavy use of zero-knowledge proofs of knowledge.
      \item Can prove equalities, inequalities, knowledge, ownership, \dots
      \item Implementations and approaches:
        \begin{description}
          \item[Identity Mixer]
            \url{https://www.research.ibm.com/labs/zurich/idemix/}
          \item[U-Prove]
            \url{http://research.microsoft.com/en-us/projects/u-prove/}
          \item[AnonPass]
            \url{https://eprint.iacr.org/2013/317}
          \item[IRMA]
            \url{https://www.irmacard.org/irma/}
        \end{description}
    \end{itemize}
  \end{example}
\end{frame}


%%%%%%%%%%%%%%%%%%%%%%


\begin{frame}[allowframebreaks]
	\small
  \printbibliography{}
\end{frame}

