\mode*

% Since this a solution template for a generic talk, very little can
% be said about how it should be structured. However, the talk length
% of between 15min and 45min and the theme suggest that you stick to
% the following rules:  

% - Exactly two or three sections (other than the summary).
% - At *most* three subsections per section.
% - Talk about 30s to 2min per frame. So there should be between about
%   15 and 30 frames, all told.


\section{User--machine authentication}

\subsection{What's the problem?}

\begin{frame}
  \begin{idea}
    \begin{itemize}
      \item A machine provides a service.
      \item A user wants to authenticate to the machine to use the service.
      \item How can the machine know the user is the right one?
    \end{itemize}
  \end{idea}

  \begin{example}
    \begin{itemize}
      \item The machine is the user's computer.
      \item The machine is the user's smartphone.
      \item The machine is the user's smartcard (\eg VISA, MasterCard).
    \end{itemize}
  \end{example}
\end{frame}

\subsection{General approaches}

\begin{frame}
  \begin{idea}[Methods for user authentication]
    \begin{itemize}
      \item Something you \emph{know}

        \pause

      \item Something you \emph{have}

        \pause

      \item Something you \emph{are}
    \end{itemize}
  \end{idea}
\end{frame}

\begin{frame}
  \begin{example}[Something you know]
    \begin{itemize}
      \item A password (PIN, swipe pattern, \etc).
      \item A cryptographic key (a longer password).
    \end{itemize}
  \end{example}
\end{frame}

\begin{frame}
  \begin{example}[Something you have]
    \begin{itemize}
      \item \Iac{PUF}.
      \item A hardware token (smartcard, embedded cryptographic key).
    \end{itemize}
  \end{example}

  \pause

  \begin{remark}
    \begin{itemize}
      \item Where's the border between something you know and something you 
        have?
      \item Do you know or have a cryptographic key?
    \end{itemize}
  \end{remark}
\end{frame}

\begin{frame}
  \begin{example}[Something you are, biometrics]
    \begin{itemize}
      \item Fingerprint
      \item Iris
      \item Gait (walking style)
      \item Handwriting (style, pressure points, \etc)
      \item Ear shape
    \end{itemize}
  \end{example}
\end{frame}

\subsection{Multi-factor authentication}

\begin{frame}
  \begin{definition}[Multi-factor authentication]
    \begin{itemize}
      \item Combine two or more methods of authentication: something you know, 
        have or are.
    \end{itemize}
  \end{definition}
\end{frame}

\begin{frame}
  \begin{example}[Single-factor authentication]
    \begin{description}
      \item[Identification] Username
      \item[Authentication] Something you know, \ie a password
    \end{description}
  \end{example}

  \pause{}

  \begin{example}[Two-factor authentication]
    \begin{description}
      \item[Identification] Username
      \item[Authentication] Something you know together with something you 
        have, \eg password and mobile phone
    \end{description}
  \end{example}
\end{frame}


%%%%%%%%%%%%%%%%%%%%%%


\begin{frame}[allowframebreaks]
	\small
  \printbibliography{}
\end{frame}

