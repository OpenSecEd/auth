\mode*

% Since this a solution template for a generic talk, very little can
% be said about how it should be structured. However, the talk length
% of between 15min and 45min and the theme suggest that you stick to
% the following rules:  

% - Exactly two or three sections (other than the summary).
% - At *most* three subsections per section.
% - Talk about 30s to 2min per frame. So there should be between about
%   15 and 30 frames, all told.


\section[Intro authentication]{Introduction to authentication}

\subsection{Identification and authentication}

\begin{frame}
  \begin{definition}[Identifier]
    \begin{itemize}
      \item An identifier is a piece of data that uniquely identifies some 
        entity.
    \end{itemize}
  \end{definition}

  \begin{example}[Identifiers]
    \begin{itemize}
      \item An email address identifies a user uniquely in the email system.
      \item A username identifies a user in some system.
      \item A passport number uniquely identifies a passport issued by 
        a country.
    \end{itemize}
  \end{example}
\end{frame}

\begin{frame}
  \begin{definition}[Authentication]
    \begin{itemize}
      \item Some entity claims some attribute of some data.
      \item \Eg identity: \enquote{identifier \(X\) identifies me}.
      \item Authentication is about verification.
      \item That entity must \emph{convince} us that its claim is true.
    \end{itemize}
  \end{definition}

  \pause{}

  \begin{exercise}[How can we authenticate \dots]
    \begin{enumerate}
      \item the claim of an email address?
      \item the claim of a username in some system?
      \item the claim of a passport number?
      \item the claim of a national identity in some country?
    \end{enumerate}
  \end{exercise}
\end{frame}

\begin{frame}
  \begin{example}[User authentication]
    \begin{description}
      \item[Identification] First you enter your username to \emph{identify} 
        yourself.

      \item[Authentication] Then you enter your password to \emph{authenticate} 
        that you are truly you.
    \end{description}
  \end{example}

  \pause{}

  \begin{exercise}
    Why does this work?
  \end{exercise}
\end{frame}

\begin{frame}
  \begin{example}
    \begin{itemize}
      \item Identity is simply an attribute.
      \item Age is another attribute.
      \item \enquote{Authorized to read document X} is also an attribute.
      \item \enquote{Is an administrator} is an attribute too.
    \end{itemize}
  \end{example}
\end{frame}

\subsection{Authenticating}

\begin{frame}
  \begin{example}[Age limits]
    \begin{itemize}
      \item Bob wants to go see a film in cinema.
      \item Bob looks very young so Alice who works there wants to have proof 
        of his age.

        \pause

      \item Show physical ID; reveals name, exact date of birth, \dots
    \end{itemize}
  \end{example}

  \pause

  \begin{exercise}
    \begin{itemize}
      \item That's a bit overkill, right?
      \item What does Alice actually need to know?
      \item In what direction must we move to achieve that?
    \end{itemize}
  \end{exercise}
\end{frame}

\begin{frame}
  \begin{block}{What Alice needs?}
    She must be convinced that Bob is older than 15.
  \end{block}

  \pause

  \begin{alertblock}{How can she learn that?}
    \begin{enumerate}
      \item She has known Bob since he was born, so she knows.

        \pause

      \item She can ask someone \emph{she trusts} who knows Bob is older than 
        15.
    \end{enumerate}
  \end{alertblock}
\end{frame}

\begin{frame}
  \begin{alertblock}{But how can she do that?}
    \begin{enumerate}
      \item The trusted person who knows Bob is with Alice.

        \pause

      \item Alice can send a picture to the other person who verifies.
        \begin{itemize}
          \item This requires an \emph{authenticated} channel.
        \end{itemize}

        \pause

      \item The trusted person made a certificate for Bob showing that he's 
        older than 15.
        \begin{itemize}
          \item Alice must be able to \emph{verify} the certificate.
          \item Bob must not be able to forge such a certificate.
          \item Bob must bring this certificate with himself everywhere.
        \end{itemize}

    \end{enumerate}
  \end{alertblock}
\end{frame}

\begin{frame}
  \begin{alertblock}{Alice interacts with the trusted person}
    \begin{itemize}
      \item Gaah, but Bob doesn't want the trusted person (his parents) to know 
        he's at the cinema right now!
      \item It's a small cinema so they'll know which film he sees if they 
        learn when he's there.
    \end{itemize}
  \end{alertblock}

  \pause

  \begin{alertblock}{Alice reads and verifies the certificate}
    \begin{itemize}
      \item Phew, she accepted the note from his parents.
      \item But now Alice learned all those embarrassing things in 
        there.
    \end{itemize}
  \end{alertblock}
\end{frame}

\subsection{DIY or trust}

\begin{frame}
  \begin{itemize}
    \item Some attributes we can verify ourselves (\ie DIY).
    \item For other attributes we rely on someone else (\ie trust).
  \end{itemize}
\end{frame}

\begin{frame}
  \begin{example}[Verify ourselves]
    \begin{itemize}
      \item I sign a note saying \enquote{Pay 10 SEK for this note}.
      \item I give you this authentication token if I owe you 10 SEK.
      \item I can verify the authenticity of the note when you claim the money.
    \end{itemize}
  \end{example}

  \pause

  \begin{example}[Trust someone else]
    \begin{itemize}
      \item I ask your age.
      \item You show me your ID card.
      \item I trust the card issuer and read your birthday.
    \end{itemize}
  \end{example}
\end{frame}

\begin{frame}
  \begin{remark}
    \begin{itemize}
      \item Both methods use \enquote{authentication tokens}.
      \item Security depends on forgeability of those.
    \end{itemize}
  \end{remark}
\end{frame}

\subsection{Time of check, time of use}

\begin{frame}
  \begin{remark}
    \begin{itemize}
      \item Whenever we authenticate a user, we do this for a purpose.

      \item When does this authentication take place in relation to when we make 
        use of it?
    \end{itemize}
  \end{remark}
\end{frame}

\begin{frame}
  \begin{example}[Bank office]
    \begin{itemize}
      \item Customer shows ID to clerk, clerk verifies account owner 
        (authentication).
      \item Clerk helps the customer.
      \item Customer leaves.
    \end{itemize}
  \end{example}
\end{frame}

\begin{frame}
  \begin{example}[Personal computers]
    \begin{itemize}
      \item User starts the computer in the morning.
      \item User logs in (authentication step).
      \item User goes for coffee.
      \item User comes back.
      \item User goes to lunch.
      \item User comes back.
      \item \dots
      \item User turns the computer off.
    \end{itemize}
  \end{example}
\end{frame}

\begin{frame}
  \begin{remark}
    \begin{itemize}
      \item The key difference is that the first (bank) has \enquote{continuous 
          authentication}, the clerk notices if the customer changes.

      \item The computer doesn't see the difference who's at the keyboard.
      \item We need continuous authentication for the computer too.
    \end{itemize}
  \end{remark}
\end{frame}

\begin{frame}
  \begin{solution}
    \begin{itemize}
      \item We can approximate that with \emph{repeated authentication}.

      \item We could also authenticate anew when we need to do something 
        requiring more privileges, and if it has been a while since last time.

      \item The computer could monitor the user's behaviour.
    \end{itemize}
  \end{solution}
\end{frame}


%%%%%%%%%%%%%%%%%%%%%%


\begin{frame}[allowframebreaks]
	\small
  \printbibliography
\end{frame}

