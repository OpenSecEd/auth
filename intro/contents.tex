\mode*

% Since this a solution template for a generic talk, very little can
% be said about how it should be structured. However, the talk length
% of between 15min and 45min and the theme suggest that you stick to
% the following rules:  

% - Exactly two or three sections (other than the summary).
% - At *most* three subsections per section.
% - Talk about 30s to 2min per frame. So there should be between about
%   15 and 30 frames, all told.


\section{Introduction}

\subsection{Identification and Authentication}

\begin{frame}
  \begin{definition}[Identifier]
    \begin{itemize}
      \item An identifier is a piece of data that uniquely identifies some 
        entity.
    \end{itemize}
  \end{definition}

  \begin{example}[Identifiers]
    \begin{itemize}
      \item An email address identifies a user uniquely in the email system.
      \item A username identifies a user in some system.
      \item A passport number uniquely identifies a passport issued by 
        a country.
    \end{itemize}
  \end{example}
\end{frame}

\begin{frame}
  \begin{definition}[Authentication]
    \begin{itemize}
      \item Some entity claims some attribute of some data.
      \item \Eg identity: \enquote{identifier \(X\) identifies me}.
      \item Authentication is about verification.
      \item That entity must \emph{convince} us that its claim is true.
    \end{itemize}
  \end{definition}

  \pause{}

  \begin{exercise}[How can we authenticate \dots]
    \begin{itemize}
      \item \dots the claim of an email address?
      \item \dots the claim of a username in some system?
      \item \dots the claim of a passport number?
      \item \dots the claim of a national identity in some country?
    \end{itemize}
  \end{exercise}
\end{frame}

\begin{frame}
  \begin{example}[User authentication]
    \begin{description}
      \item[Identification] First you enter your username to \emph{identify} 
        yourself.

      \item[Authentication] Then you enter your password to \emph{authenticate} 
        that you are truly you.
    \end{description}
  \end{example}

  \pause{}

  \begin{exercise}
    Why does this work?
  \end{exercise}
\end{frame}


%%%%%%%%%%%%%%%%%%%%%%


\begin{frame}[allowframebreaks]
	\small
  \printbibliography{}
\end{frame}

